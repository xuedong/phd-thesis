% !TEX root = ../Chapter5.tex
\section{\HCT{} under Local Smoothness w.r.t. \texorpdfstring{$\cP$}{}}\label{sec:gpo.hct}

Analyzing \HOO under Assumption~\ref{ass1} is not trivial. A key lemma in the analysis of \HOO  (Lemma~3 by \citealt{bubeck2011pure}) that controls the variance of near-optimal cells \emph{is not true} under local smoothness assumptions as Assumption~\ref{ass1}. Indeed, \HOO could induce a very deep covering tree,  while producing too many nodes that are neither near-optimal nor sub-optimal. The concept of near-optimal and sub-optimal nodes is then characterized by the \emph{sub-optimality gap} of each node which measures the distance between the local maximum of the node and the global maximum. Intuitively, nodes that are neither near-optimal nor sub-optimal represent the nodes of whom the sub-optimality gap is neither too large nor too small. To control the regret due to these nodes,~\cite{bubeck2011pure} use global smoothness (weakly Lipschitz) assumption. Assumption~\ref{ass1} is weaker, only local, and does not offer such comfort. If we want to control the regret due to these nodes without Lemma~3 of \citet{bubeck2011pure}, one possible way is to control the depth of the covering tree to ensure that we do not have too many of them. In particular, another algorithm known as \HCT \citep{azar2014online} implies a controlled depth of the tree which allows it to be analyzed under Assumption~\ref{ass1} as opposed to \HOO. We now give a brief description of \HCT and present a new analysis of it.

\subsection{Description of \HCT}
\begin{algorithm}[ht]
\centering
\caption{High confidence tree (\HCT, \citealt{azar2014online})}
\label{alg:hct}
\begin{algorithmic}[1]
    \State {\bfseries Input:} $K$, $\nu>0$, $\rho\in(0,1)$, $c>0$, tree partitioning $\{\mathcal{P}_{h,i}\}$, confidence $\delta$
    \State {\bfseries Initialization:} $\mathcal{T}_1 \gets \{(0,1),(1,1),\ldots,(1,K)\}$, $U_{1,1}(1) \gets \cdots \gets U_{1,K}(1) \gets +\infty$
    \For{$t \gets 1 \ldots n$}
        \If{$t=t^+$}
            \For{$(h,i)\in\mathcal{T}_t$}
                \State $U_{h,i}(t) \gets \hat{\mu}_{h,i}(t) + \nu\rho^{h} + c\sqrt{\frac{\operatorname{log}\left(1/\tilde{\delta}(t^+)\right)}{T_{h,i}(t)}}$
            \EndFor
            \State \texttt{UpdateBackward}$(\cT_t, t)$
        \EndIf
        \State $(h_t,i_t),P_t \gets$ \texttt{OptTraverse}($\cT_t, t$)
        \State \texttt{evaluate} $x_{h_t,i_t}$ and obtain $r_t$
        \State $T_{h_t,i_t}(t) \gets T_{h_t,i_t}(t)+1$
        \State \texttt{update} $\hat{\mu}_{h_t,i_t}(t)$
        \State $U_{h_t,i_t}(t) \gets \hat{\mu}_{h_t,i_t}(t) + \nu\rho^{h_t} + c\sqrt{\frac{\operatorname{log}\left(1/\tilde{\delta}(t^+)\right)}{T_{h_t,i_t}(t)}}$
        \State \texttt{UpdateBackward}($P_t, t$)
        \State $\tau_{h_t}(t) \gets \ceil{\frac{c^2\operatorname{log}(1/\tilde{\delta}(t^+))}{\nu^2}\rho^{-2h_t}}$
        \If{$T_{h_t,i_t}(t)\geq\tau_{h_t}(t)$ and $(h_t,i_t)$ is a leaf}
            \State \texttt{Expand}($(h_t,i_t)$)
        \EndIf
    \EndFor
\end{algorithmic}
\end{algorithm}

\begin{algorithm}[ht]
\centering
\caption{\texttt{OptTraverse}
\label{alg:hct_opt}}
\begin{algorithmic}[1]
    \State {\bfseries Input:} a tree $\cT$, round $t$
    \State {\bfseries Initialization:} $(h,i)\leftarrow (0,1)$; $P\leftarrow \{(0,1)\}$; $T_{0,1}(t)=\tau_0(t)=1$
    \While{$(h,i)$ is not a leaf of $\cT$ and $T_{h,i}(t)\geq\tau_h(t)$}
        \State $j \gets \underset{j\in\{0,\ldots,K-1\}}{\argmax} \left\{B_{h+1,Ki-j}(t)\right\}$
        \State $(h,i) \gets (h+1,Ki-j)$
        \State $P\leftarrow P\cup\{(h,i)\}$
    \EndWhile
    \State {\bfseries Return} $(h,i)$ and $P$
\end{algorithmic}
\end{algorithm}

\begin{algorithm}[ht]
\centering
\caption{\texttt{UpdateBackward} \label{alg:hct_update}}
\begin{algorithmic}[1]
    \State {\bfseries Input:} a tree $\cT$, round $t$ \\ \hfil \emph{note that $P_t$ can also be considered as a tree, thus input of this function}
    \For{$(h,i)\in\cT$ backward from each leaf of $\cT$}
        \If{$(h,i)$ is a leaf of $\cT$}
            \State $B_{h,i}(t) \gets U_{h,i}(t)$
        \Else 
            \State $B_{h,i}(t) \gets \min\left\{U_{h,i}(t),\underset{j\in\{0,\ldots,K-1\}}{\max} \left\{B_{h+1,Ki-j}(t)\right\} \right\}$
        \EndIf
    \EndFor
\end{algorithmic}
\end{algorithm}

The pseudocode of \HCT (Algorithm~\ref{alg:hct}) and two detailed snippets (Algorithm~\ref{alg:hct_opt} and Algorithm~\ref{alg:hct_update}) describe the process of traversing the covering tree.
The algorithm stores a finite subtree $\cT_t$ at each round~$t$ which is initialized by $\cT_0 = \{(0,1)\}$. Each cell is associated with a representative point $x_{h,i}$ and the algorithm keeps track of some statistics regarding this point. One of these statistics is the empirical mean reward $\hat{\mu}_{h,i}(t)$ which is the average on the first $T_{h,i}(t)$ rewards received when querying $x_{h,i}$.
%\[
%	\hat{\mu}_{h,i}(t) \triangleq \frac{1}{T_{h,i}(t)}\sum_{s=1}^{T_{h,i}(t)}r_{(h,i),s},
%\]
The \HCT algorithm also keeps track of an upper confidence bound $U$-value for the cell $(h,i)$,
\[
	U_{h,i}(t) \triangleq \hat{\mu}_{h,i}(t) + \nu\rho^{h} + c\sqrt{\frac{\operatorname{log}(1/\tilde{\delta}(t^+))}{T_{h,i}(t)}}\CommaBin
\]
where $t^+\triangleq2^{\lceil \operatorname{log}_2(t) \rceil}$, $\tilde{\delta}(t)\triangleq\min\{c_1\delta/t,1/2\}$, and its corresponding $B$-value,
\begin{equation*}
	B_{h,i}(t) \triangleq
	\left\{ \begin{array}{ll}
				\min\left\{U_{h,i}(t),\underset{j\in\{0,\ldots,K-1\}}{\max} \left\{B_{h+1,Ki-j}(t)\right\} \right\} & \operatorname{if}~(h,i) \text{ is an internal node,}\\
				U_{h,i}(t) & \operatorname{otherwise,}
			\end{array}\right.
\end{equation*}
which is designed to be a tighter upper confidence bound than the $U$-value. Here, $c$ and $c_1$ are two constants, and $\nu\rho^h$ represents the \emph{resolution}\footnote{The term \emph{resolution} refers to the maximum variation in the cell. If it is too large, then we need to shrink the volume, thus increase the resolution.} of the region $\cP_{h,i}$. Observe that $U_{h,i}(t)$ and $B_{h,i}(t)$ are not updated at every round, but are constant on time intervals of the form $[2^k,2^{k+1})$.

At each round $t$, the algorithm traverses the current covering tree along an \emph{optimistic path}~$P_t$ before choosing a point (\texttt{OptTraverse} function). This optimistic path~$P_t$ is obtained by repeatedly selecting cells that have a larger $B$-value until a leaf or a node that is sampled less than a certain number of times is reached. If a leaf is reached, then this leaf is sampled and expanded (i.e., we split the leaf into $K$ equal-sized regions and initialize their $U$-values to $+\infty$); otherwise, the node that is not sampled enough is re-sampled. All the $B$-values along the optimistic path are then updated backwardly from the current node to the root (\texttt{UpdateBackward} function). More precisely, \HCT{} samples one node a certain number of times $\tau_h(t)$ in order to sufficiently reduce the uncertainty before expanding it. Hence, $\tau_h(t)$ is defined such that the uncertainty over the rewards in $\mathcal{P}_{h,i}$ is roughly equal to the resolution of the node,
\[
	\tau_h(t) \triangleq \ceil{\frac{c^2\operatorname{log}(1/\tilde{\delta}(t^+))}{\nu^2}\rho^{-2h}}.
\]

\subsection{Analysis of \HCT \textbf{under a local \emph{metricless} assumption}}\label{sec:gpo.analysis}
We now state our main theorem. We prove that \HCT achieves an expected regret bound under Assumption~\ref{ass1} which matches the regret bound given by~\cite{azar2014online} up to constants. Moreover, compared to that result, the near-optimality dimension $d$ featured in Theorem~\ref{thm:hct} is the one of Definition~\ref{defNearOpt} that is defined with respect to the partitioning and not with respect to a metric. For a fixed budget $n$, we introduce the notation $\HCT{}(\nu,\rho)$ to refer to the instantiation of $\HCT{}$ parameterized by $\nu$, $\rho$, $c=2\sqrt{1/(1-\rho)}$ and $\delta = 1/n$.

\begin{restatable}{theorem}{restathm}\label{thm:hct}
	Assume that function~$f$ satisfies Assumption~\ref{ass1}. Then, setting $\delta \eqdef 1/n$, the cumulative regret of ${\HCT}(\nu,\rho)$ after $n$ function evaluations is upper bounded as 
\[
	\mathbb{E}[R_n^{\HCT{}(\nu,\rho)}] \leq \alpha C (\operatorname{log}n)^{1/(d(\nu,C,\rho)+2)}n^{(d(\nu,C,\rho)+1)/(d(\nu,C,\rho)+2)}\!,
\]
where $\alpha$ is a numerical constant and $C$ is the constant associated to $d(\nu,C,\rho)$.
%\in\cC(\nu,\rho)$.
\end{restatable}\noindent
As a consequence, by simply applying  the recommendation strategy that follows the distribution of previous plays, we get the following simple-regret bound.
\begin{corollary}\label{col}
The simple regret  of {\HCT} after $n$ function evaluations under Assumption~\ref{ass1} satisfies
\[
	\mathbb{E}[S_n^{\HCT{}(\nu,\rho)}] \leq \alpha C(\operatorname{log}n)^{1/(d(\nu,C,\rho)+2)}n^{-1/(d(\nu,C,\rho)+2)}.
\]
\end{corollary}
We now sketch the proof. The full proof follows the analysis of~\cite{azar2014online} and is detailed in Appendix~\ref{app:gpo.hct}.
As  mentioned above, \HCT has a controlled depth. Indeed, given the threshold $\tau_h(t)$ required at depth $h$, in Section~\ref{proof:lemma_depth}, we prove that the depth of the covering tree is bounded as stated in the following lemma.
\begin{restatable}{lemma}{restalemmadepth}\label{lemma_depth}
The depth of the covering tree produced by \HCT after $n$ function evaluations satisfies \[H(n) \leq H_{\max}(n) \triangleq \ceil{\frac{1}{2(1-\rho)} \operatorname{log}\left( \frac{n\nu^2}{c^2\rho^2} \right)}\!\cdot\]
\end{restatable}
\noindent
Defining the mean reward $\mu_{h,i} \triangleq f(x_{h,i})$, we introduce a favorable event under which the mean reward of all expanded nodes is within a confidence interval,
\[
	\xi_t \triangleq \left\{ \forall (h,i)\in\mathcal{L}_t,  |\hat{\mu}_{h,i}(t) - \mu_{h,i}| \leq c\sqrt{\operatorname{log}(1/\tilde{\delta}(t))/T_{h,i}(t)} \right\}\CommaBin
\]
where $\cL_t$ is the set of all possible nodes in trees of maximum depth $H_{\max}(t)$.

We split the regret into two parts depending on whether $\xi_t$ holds or not.
In Appendix~\ref{proof:lemma_failing}, we prove that the failing confidence term is with high probability bounded by $\sqrt{n}$. In the case when  $\xi_t$ holds, we bound the regret in Appendix~\ref{proof:thm}
by treating separately the two parts, $\Delta_{h_t,i_t}$ and $\hat{\Delta}_t$, of the instantaneous regret $\Delta_t$,
\[
\Delta_t \eqdef f^\star - r_t = f^\star - f(x_{h_t,i_t}) + f(x_{h_t,i_t}) - r_t = \Delta_{h_t,i_t} + \hat{\Delta}_t.
\]
Next, we bound $ \hat{\Delta}_t$ by Azuma-Hoeffding concentration inequality \citep{azuma1967}.
Then, we bound $ \Delta_{h_t,i_t}$  with the help of the following lemma, which is the major difference compared to the original \HCT analysis by~\cite{azar2014online}. In particular, the lemma states that
if Assumption~\ref{ass1} is verified then
 $f^\star$ is upper-bounded by the $U$-value of any optimal node.
\begin{lemma}\label{upper}
	Under Assumption~\ref{ass1} and under event $\xi_t$, we have that for any optimal node $(h^\star,i^\star)$, $U_{h^\star,i^\star}(t)$ is an upper bound on $f^\star$.
\end{lemma}

\begin{proof}
Since $t^+ \geq t$, we have
\[
	U_{h^\star,i^\star}(t) \eqdef \hat{\mu}_{h^\star,i^\star}(t) + \nu\rho^{h^\star} + c\sqrt{\frac{\operatorname{log}(1/\tilde{\delta}(t^+))}{T_{h^\star,i^\star}(t)}} \geq \hat{\mu}_{h^\star,i^\star}(t) + \nu\rho^{h^\star} + c\sqrt{\frac{\operatorname{log}(1/\tilde{\delta}(t))}{T_{h^\star,i^\star}(t)}}\cdot
\]
Moreover, as we are under event $\xi_t$, we also have
\[
	\hat{\mu}_{h^\star,i^\star}(t) + c\sqrt{\frac{\operatorname{log}(1/\tilde{\delta}(t))}{T_{h^\star,i^\star}(t)}} \geq f\left(x_{h^\star,i^\star}\right).
\]
Therefore, $U_{h^\star,i^\star}(t) \geq f(x_{h^\star,i^\star}) + \nu\rho^{h^\star} \geq f^\star$.
\end{proof}
\noindent With the help of Lemma~\ref{upper} (see Step 2 in Appendix~\ref{proof:thm}), we can then upper bound $\Delta_{h_t,i_t}$ as
\[
\Delta_{h_t,i_t} \leq 3c\sqrt{\frac{\operatorname{log}(2/\tilde{\delta}(t))}{T_{h_t,i_t}(t)}}\cdot%\CommaBin
\]
To bound the total regret of the all nodes selected, we divide them
into two categories, depending on whether their depth is smaller or equal than $\bar H$
(to be optimized later) or not.

For the nodes in depths $h \leq \bar H$, we  use
Lemma~\ref{upper} again, now to show that \texttt{OptTraverse} only selects nodes
 %\begin{equation*} %\label{eq14}
%\Delta_{h_t^p,i_t^p} \leq 3\nu\rho^{h_t^p},
%\end{equation*}
that have a parent which is $(3\nu\rho^{h_t-1})$-optimal.
For the nodes for which  $h > \bar H$, we bound the regret using the
selection rule of \HCT.


The sums of the regrets from the two categories are proportional
and inversely proportional to an increasing function of $\bar H$.
By finding the value of $\bar H$ for which the sum of the two
terms reaches its minimum and adding the regret coming
from the situations where the favorable event does not hold,
gives us the following cumulative regret for \HCT: With probability $1-\delta$,
\[
	R_n^{\HCT(\nu,\rho)} \leq \cO\left( (\operatorname{log}(n/\delta))^{1/(d(\nu,C,\rho)+2)}n^{(d(\nu,C,\rho)+1)/(d(\nu,C,\rho)+2)} \right).
\]
However, the analysis of \POO requires a bound on the expected regret of the underlying subroutine. For that purpose, we simply set $\delta \eqdef 1/n$ and that gives us the statement of Theorem~\ref{thm:hct}, and consequently Corollary~\ref{col}.
