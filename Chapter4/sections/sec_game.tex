%!TEX root = ../Chapter3.tex
\section{A Gamified Algorithm}\label{sec:lgc.game}

We present two asymptotically optimal algorithms for the general pure-exploration problem. We also make the additional assumption that the set of parameter is bounded, that is we know $M>0$ such that for all $\theta\in\cM,\, \normm{\theta}\leq M$. This assumption is shared by most of the works on %\todo{rather usual? or used everywhere? Evrywhere i would say since they target minimax regret}
linear bandits (e.g. \citealt{abbasi-yadkori2011linear, soare2014linear}).

We provide a bounded version of BAI in Section~\ref{app:lgc.examples} where we add the assumption that the parameter set $\cM$ is bounded. In particular we show how it affects the lower bound of Theorem~\ref{th:lb_genral}: the characteristic time $\Tstar(\theta)$ is reduced (or equivalently $\Tstar(\theta)^{-1}$ increases). This is not surprising since we add a new constraint in the optimization problem. This means that the algorithm should stop earlier. The counterpart of this improvement is that it is often difficult to compute the best response for nature. Indeed, for example, in BAI, there is an explicit expression of the best response, see Section~\ref{app:lgc.examples}. When the constraint $\normm{\lambda}\leq M$ is added there is no explicit expression anymore and one needs to solve an uni-dimensional optimization problem, see Lemma~\ref{lem:lagrange_alternative}. To devise an asymptotically optimal algorithm without the boundedness assumption remains an open problem.


% , i.e for $i\in\cI$ and $w\in\interior{\Sigma_A}$ in the interior of the probability simplex of dimension $A-1$, find $\lambda^i$ such that
% \[
% \lambda^i \in \argmin_{\lambda\neg i}\normm{\theta - \lambda }_{V_{w}}\,.
% \]
% For example, in BAI, there is an explicit expression of the best response. Noting that, for $\astar\in\cA$, we have
% \[
% \min_{\lambda\neg \astar}\normm{\theta - \lambda }_{V_{w}} = \min_{a\neq \astar}\min_{\langle\lambda,a-\astar\rangle>0} \normm{\theta - \lambda }_{V_{w}}\,,
% \]
% the best response is then
% \begin{align}
% \lambda^{b} &= \theta - \frac{\max(\langle\theta,\astar-b \rangle,0) }{\normm{\astar-b}_{V_w^{-1}}^2} V_w^{-1}(\astar - b) \label{eq:best_response_bai}
% \\
% \text{for }b&\in\argmin_{a\neq \astar}\normm{\theta - \lambda^{a} }_{V_{w}}\nonumber.
% \end{align}
% When the constraint $\normm{\lambda}\leq M$ is added there is no explicit expression anymore and one needs to solve an uni-dimensional optimization problem, see Lemma~\ref{lem:lagrange_alternative}.


%Note that in the proof of Theorem~\ref{thm:sample_complexity} we only use two times the boundedness assumption, first in the definition of the threshold $\beta(t,\delta)$ (see Theorem~\ref{th:confidence_beta}) to handle the bias induced by the regularization. Second, since the regret of AdaHedge is proportional to the maximum of the upper confidence bounds $U_s^{i,a}$, we need to ensure that they are bounded.

% And there exists pathological pure exploration problems where even the quantity uppper bounded by $U_s^{i,a}$, namely $\normm{\theta-\lambda}_{aa^\top}^2$ where $\lambda\in\argmin_{\lambda'\in\neg i}\normm{\theta -\lambda}$ is unbounded\todo{I don't understand what this means.}. See for example Appendix~G.3 by \citet{menard2019lma}.


We describe primarily \LGC, detailed in Algorithm~\ref{alg:lgc}. The principle behind \LG, detailed in Algorithm~\ref{alg:lg}, is similar and significant differences will be highlighted.


\subsection{Notations}
\paragraph{Counts.} At each round $t$ the algorithms will play an arm $a_t$ and choose (fictitiously) an answer $i_t$. We denote by $N_t^{a,i} \eqdef\sum_{s=1}^t \ind_{\{(a_t,i_t)=(a,i)\}}$ the number of times the pair $(a,i)$ is chosen up to and including time $t$, and by $N_t^a =\sum_{i\in\cI} N_s^{a,i}$ and $N_t^i =\sum_{a\in\cI} N_s^{a,i}$ the partial sums. The vectors of counts at time $t$ is denoted by $N_t \eqdef (N_t^a)_{a\in\cA}$
and when it is clear from the context we will also denote by $N_t^a = (N_t^{a,i})_{i\in\cI}$ and $N_t^i = (N_t^{a,i})_{i\in\cI}$ the vectors of partial counts.

\paragraph{Regularized least square estimator.} We fix a regularization parameter $\eta > 0$. The regularized least square estimator for the parameter $\theta\in \mathcal M$ at time $t$ is
\[
\htheta_{t} = (V_{N_t} + \eta I_d)^{-1} \sum_{s=1}^t Y_s a_s\,,
\]
where $I_d$ is the identity matrix. By convention $\htheta_0 = 0$.


\subsection{Algorithms}

\begin{algorithm}[tb]
\centering
\caption{\LG}
\label{alg:lg}
\begin{algorithmic}[1]
    \State {\bfseries Input:} Agent learners for each answers $(\cL^i_w)_{i\in\cI}$, threshold $\beta(\cdot,\delta)$
    \For{t = 1 \ldots}
        \State \textit{// Stopping rule}
        \If{{$\max_{i\in\cI}\inf_{\lambda\in\neg i} \frac{1}{2}\normm{\htheta_{t-1}-\lambda}^2_{V_{N_{t-1}}}\geq\! \beta(t-1,\delta)$}}
            \State \texttt{stop} 
            \State {\bfseries Return} $\hi = \istar(\hat{\theta}_{t-1})$
        \EndIf
        \State \textit{// Best answer}
        \State $i_{t} = \istar(\hat{\theta}_{t-1})$
        \State \textit{// Agent plays first}
        \State \texttt{get} $w_{t}$ \texttt{from} $\cL^{i_t}_w$
        \State \texttt{update} $W_{t}=W_{t-1}+w_{t}$
        \State \textit{// Best response for the nature}
        \State $\lambda_{t} \in \argmin_{\lambda\in\neg i_t}\normm{\htheta_{t-1}-\lambda}^2_{V_{w_{t}}}$
        \State \textit{// Feed optimistic gains}
        \State \texttt{feed learner} $\cL_w^{i_{t}}$ \texttt{with} $g_{t}(w) = \sum_{a\in\cA}w^a U_t^a/2$
        \State \textit{// Track the weights}
        \State \texttt{pull} $a_{t}\in \argmin_{a\in \mathcal A } N_{t-1}^{a} - W_{t}^{a}$
   \EndFor
\end{algorithmic}
\end{algorithm}

\begin{algorithm}[tb]
\centering
\caption{\LGC}
\label{alg:lgc}
\begin{algorithmic}[1]
    \State {\bfseries Input:} Agent learner $\cL_{\tw}$, threshold $\beta(\cdot,\delta)$
    \For{t = 1 \ldots}
        \State \textit{// Stopping rule}
        \If{ {\small $\max_{i\in \cI} \inf_{\lambda\in\neg i} \frac{1}{2}\normm{\htheta_{t-1}-\lambda}^2_{V_{N_{t-1}}}\geq \beta(t-1,\delta)$}}
            \State {\bfseries stop} and {\bfseries return} $\hi = \istar(\hat{\theta}_{t-1})$. 
        \EndIf
        \State \textit{// Agent plays first}
        \State \texttt{get} $\tw_{t}$ \texttt{from} $\cL_{\tw}$ 
        \State \texttt{update} $\tW_{t}=\tW_{t-1}+\tw_{t}$
        \State \textit{// Best response for the nature}
        \State $\forall i\in\cI$, $\tlambda^i_{t} \in \argmin_{\lambda\in\neg i}\normm{\htheta_{t-1}-\lambda}^2_{V_{\tw^i_{t}}}$
        \State \textit{// Feed optimistic gains}
        \State \texttt{feed learner} $\cL_{\tw}$ \texttt{with} {$g_{t}(\tw) =\sum_{(a,i)\in\cA\times\cI} \tw^{a,i} U_t^{a,i}/2$}
        \State \textit{// Track the weights}
        \State \texttt{pull} $(a_{t},i_{t})\in \argmin_{(a,i)\in \mathcal A \times \mathcal I} N_{t-1}^{a,i} - \tW_{t}^{a,i}$
    \EndFor
\end{algorithmic}
\end{algorithm}

\paragraph{Stopping rule.}
Our algorithms share the same stopping rule. Following \citet{garivier2016tracknstop}, our algorithms stop if a generalized likelihood ratio exceeds a threshold. It stops if
\begin{align}
\label{eq:def_chernoff_stopping}
\max_{i\in \mathcal I} \inf_{\lambda_i \in \neg i}\frac{1}{2}\Vert \tilde{\theta}_t - \lambda_i \Vert^2_{V_{N_t}}
> \beta(t,\delta)\,,
\end{align}
and return {\small $\displaystyle \istar_t\! \in\! \argmax_{i\in \mathcal I}\! \inf_{\lambda_i \in \neg i}\Vert \tilde{\theta}_t - \lambda_i \Vert^2_{V_{N_t}}/2$}.
This stopping and decision rules ensures that the algorithms \LG and \LGC are $\delta$-correct regardless of the sampling rule used, see lemma below\footnote{The fact that $\tau_\delta <+\infty$ is a consequence of our analysis, see Appendix~\ref{app:lgc.proof}.} proved in Appendix~\ref{app:lgc.concentration}.
\begin{lemma}
\label{lem:chernoff_stopping rule_pac}
Regardless of the sampling rule, the stopping rule~\eqref{eq:def_chernoff_stopping} with the threshold
{\small\begin{equation} \label{eq:def_beta}
\beta(t,\delta) =\left( \sqrt{\log\!\left( \frac{1}{\delta}\right)+\frac{d}{2}\log\!\left(1+\frac{t L^2}{\eta d} \right)} +\sqrt{\frac{\eta}{2}}M\right)^2\!\!\!,
\end{equation}}
satisfy $ \P_{\theta}\big(\tau_{\delta} < \infty \wedge \istar_t \neq \istar(\theta)\big) \leq \delta$.
\end{lemma}
% This stopping and decision rules ensures that the algorithm is $\delta$-correct regardless of the sampling rule used (see \citealt{garivier2016tracknstop} for a proof), hence the following theorem is immediate.
% \begin{theorem}
% Algorithms~\ref{alg:lg} and~\ref{alg:lgc} are $\delta$-correct.
% \end{theorem}
Our contribution is a sampling rule that minimizes the sample complexity when combined with these stopping and decision rules.
We now explain our sampling strategy to ensure that the stopping threshold is reached as soon as possible.

\paragraph{Saddle point computation.}
Suppose in this paragraph, for simplicity, that the parameter vector $\theta$ is known. By the definition of the stopping rule and the generalized likelihood ratio, as long as the algorithm does not stop,
\begin{align*}
\beta(t,\delta)
&\ge \inf_{\lambda\in \neg i^\star(\theta)} \sum_{a\in \cA} N_t^a \Vert \theta - \lambda \Vert^2_{a a^\top}/2 \: .
\end{align*}
If we manage to have $N_t \approx t w^\star(\theta)$ (the optimal pulling proportions at $\theta$), then this leads to $\beta(t,\delta) \ge t T^\star(\theta)^{-1}$ and, solving that equation, we have asymptotic optimality.

Since there is only one correct answer, the parameter $\theta$ belongs to all sets $\neg i$ for $i\neq \istar(\theta)$. Hence
\begin{align*}
&\inf_{\lambda\in \neg i^\star(\theta)}\frac{1}{2} \sum_{a\in \cA} N_t^a \Vert \theta - \lambda \Vert^2_{a a^\top}
\\&\geq \inf_{\tlambda_t\in \prod_i (\neg i)}\frac{1}{2}\sum_{(a,i)\in \cA\times\cI}\!\!\!\!\! N_t^{a,i} \Vert \theta - \tlambda^i \Vert^2_{a a^\top}.
\end{align*} Introducing the sum removes the dependence in the unknown $i^*(\theta)$. \LGC then uses an agent playing weights w in $\Sigma_{\cA\cI}$. \LG does not use that sum over answers, but uses a guess for $\istar(\theta)$. Its analysis involves proving that the guess is wrong only finitely many times in expectation.

Our sampling rule implements the lower bound game between an agent, playing at each stage $s$ a weight vector $\tw_s$ in the probability simplex $\Sigma_{A\times I}$, and nature, who computes at each stage a point $\lambda_s^i \in \neg i$ for all $i\in \mathcal I$. We additionally ensure that $N_t^{a,i} \approx \sum_{s=1}^t \tw_s^{a,i}$. Suppose that the sampling rule is such that at stage $t$, a $\varepsilon_t$-approximate saddle point is reached for the lower bound game, see Lemma~\ref{lem:sion_convexify}. That is,
\begin{align*}
&\inf_{\tlambda \in \prod_{i}(\neg i)} \sum_{s=1}^t \sum_{(a,i)\in \cA\times\cI} \tw_s^{i,a} \Vert \theta - \tlambda^i \Vert^2_{a a^\top}/2 +\varepsilon_t
\\
&\ge \sum_{s=1}^t \sum_{(a,i)\in \cA\times\cI} \tw_s^{i,a} \Vert \theta - \tlambda_s^i \Vert^2_{a a^\top}/2
\\
&\ge\max_{(a,i)\in \cA\times\cI} \sum_{s=1}^t \Vert \theta - \tlambda_s^i \Vert^2_{a a^\top}/2 - \varepsilon_t \: .
\end{align*}
Then if the algorithm did not stop, it verifies, using Lemma~\ref{lem:sion_convexify},
\begin{align*}
\beta(t,\delta)
&\ge t \max_{(a,i)\in \cA\times\cI} \frac{1}{t}\sum_{s=1}^t \Vert \theta - \tlambda_s^i \Vert^2_{a a^\top}/2 - 2\varepsilon_t
\\
&\ge t \inf_{\tq \in \mathcal \prod_{i\in \mathcal I}\cP(\neg i)} \! \max_{(a,i)\in \cA\times\cI} \!\!\!\! \mathbb{E}_{\lambda^i\sim q^i}\Vert \theta \! - \! \tlambda^i \Vert^2_{a a^\top}/2 \! - \! 2\varepsilon_t\\
&= t T^\star(\theta)^{-1} - 2 \varepsilon_t \: .
\end{align*}
Solving that equation, we get asymptotically the wanted $t\lesssim T^\star(\theta) \log(1/\delta)$.

We implement the saddle point algorithm by using AdaHedge for the agent (a regret minimizing algorithm of the exponential weights family), and using best-response for the nature, which plays after the agent. Precisely the learner $\cL_w$ for \LGC is AdaHedge on $\Sigma_{AI}$ with the gains
\[
g_t^\theta(\tw) = \frac{1}{2} \sum_{(a,i)\in\cA\times\cI}  \tw^{a,i} \Vert \theta - \tlambda_s^i \Vert^2_{a a^\top}\,.
\]
 Whereas \LG uses $I$ learners $\cL_w^i$, one for each possible guess of $\istar(\theta)$ with the gains. For $i\in\cI$, the learner $\cL_w^i$ is also AdaHedge but only on $\Sigma_A$ with the gains (when the guess is $i$)
 \[
 g_t^\theta(w) = \frac{1}{2} \sum_{a\in\cA}  w^{a} \Vert \theta - \lambda_s^i \Vert^2_{a a^\top}\,.
 \]
 $\varepsilon_t$ is then the sum of the regrets of the two players. Best-response has regret 0, while the regret of AdaHedge is $O(\sqrt{t})$ for bounded gains, as seen in the following lemma, taken from \citet{derooij2014hedge}.
%\todo{define the learners $\cL_w^i$ here}
\begin{lemma}\label{lem:adahedge}
On the online learning problem with $K$ arms and gains $g_s(w) = \sum_{k\in[K]} w^k  U_s^k$ for $s\in[t]$, AdaHedge, predicting $(w_s)_{s\in[t]}$, has regret
\begin{align*}
%R_T &\le 2\sqrt{(\sigma L_T - \frac{L_T^2}{T})\log(K)} + \sigma(2+\log(K)16/3) \: ,\\
R_t&:= \max_{w\in\Sigma_K}\sum_{s=1}^t g_s(w) -g_s(w_s) \\
&\le 2\sigma\sqrt{t\log(K)} + 16\sigma(2+\log(K)/3) \: ,\\
\text{where }
\sigma &= \max_{s\le t}  (\max_{k\in[K]}U_s^{k}- \min_{k\in[K]}U_s^{k}) \:.\\
%L_T &= \sum_{t=1}^T (\max_{k\in[K]}U_t^{k} - U_t^{\star}) \le T \sigma \: .
\end{align*}
\end{lemma}
Other combinations of algorithms are possible, as long as the sum of their regrets is sufficiently small. At each stage $t\in\NN$, both algorithms advance only by one iteration and as time progresses, the quality of the saddle point approximation improves. This is in contrast with \Track \cite{garivier2016tracknstop}, in which an exact saddle point is computed at each stage, at a potentially much greater computational cost.

\paragraph{Optimism.} The above saddle point argument would be correct for a known game, while our algorithm is confronted to a game depending on the unknown parameter $\theta$. Following a long tradition of stochastic bandit algorithms, we use the principle of Optimism in Face of Uncertainty. Given an estimate $\hat{\theta}_{t-1}$, we compute upper bounds for the gain of the agent at $\theta$, and feed these optimistic gains to the learner. Precisely, given the best response $\lambda_t^i \in \neg i$ we define,
\begin{align*}
U_t^{a,i} =\left\{
\begin{array}{ll}
\max_{\xi} \quad & \min\big(\Vert \xi - \lambda_t^i \Vert^2_{a a^\top},4L^2M^2\big)\\
\text{s.t.}\quad & \Vert \hat{\theta}_{t-1} - \xi \Vert^2_{V_{N_{t-1}}+\eta I_d} \le 2h(t)
\end{array}
\right. \: ,
\end{align*}
where $h(t)=\beta(t, 1/t^3)$ is some exploration function. We clipped the values, using that $\cM$ and $\cA$ are bounded to ensure bounded gains for the learners. Under the event that the true parameter verifies $\Vert \hat{\theta}_{t-1} - \theta \Vert^2 \le 2 h(t)$, this is indeed an optimistic estimate of $\Vert \theta - \lambda_t^i \Vert^2_{a a^\top}$. Note that $U_t^{a,i}$ has a closed form expression, see Appendix~\ref{app:lgc.proof}. The optimistic gain is then, for \LGC (see Algorithm~\ref{alg:lg} for the one of \LG),
\[
g_t(\tw) = \frac{1}{2} \sum_{(a,i)\in\cA\times\cI}  \tw^{a,i} U_t^{a,i}\,.
\]


\paragraph{Tracking.} In both Algorithm~\ref{alg:lg} and~\ref{alg:lgc}, the agent plays weight vectors in a simplex. Since the bandit procedure allows only to pull one arm at each stage, our algorithm needs a procedure to transcribe weights into pulls. This is what we call tracking, following \citet{garivier2016tracknstop}. The choice of arm (or arm and answer) is
\begin{align*}
a_{t+1}          &\in \argmin_{a\in \mathcal A } N_{t}^{a} - W_{t+1}^{a} & \text{ for Algorithm~\ref{alg:lg},}\\
(a_{t+1},i_{t+1})&\in \argmin_{(a,i)\in \mathcal A \times \mathcal I } N_{t}^{a,i} - \tW_{t+1}^{a,i} & \text{ for Algorithm~\ref{alg:lgc}.}
\end{align*}
This procedure guarantees that for all $t\in\NN, u \in \mathcal U$, with $\mathcal U = \mathcal A$ (resp. $\mathcal U =\cI\times\cA$) for Algo.~\ref{alg:lg} (resp. Algo.~\ref{alg:lgc}), $- \log (|\mathcal U|) \le N_t^{u} - W_t^{u} \le 1$. That result is due to~\citet{degenne2020structure}.

\begin{theorem}\label{thm:sample_complexity}
For a regularization parameter\footnote{This condition is a simple technical trick to simplify the analysis. An $\eta$ independent of $A$,$L$,$M$ will lead to the same results up to minor adaptations of the proof.} $\eta \geq 2(1+\log(A))AL^2+M^2$, for the threshold $\beta(t,\delta)$ given by~\eqref{eq:def_beta}, for an exploration function $h(t)=\beta(t,1/t^3)$, \LG and \LGC are $\delta$-correct and asymptotically optimal. That is, they verify for all $\theta\in \cM$,
\begin{align*}
\limsup_{\delta\to 0}\frac{\mathbb{E}_\theta[\tau_\delta]}{ \log 1/\delta} \le \Tstar(\theta) \: .
\end{align*}
\end{theorem}
The main ideas used in the proof are explained above. The full proof is in Appendix~\ref{app:lgc.proof} with finite $\delta$ upper bounds.
