\section*{\centering R\'esum\'e}

Dans cette th\`ese, nous \'etudions le probl\`eme d'optimisation s\'equentielle dans des enviro\-nnements stochastiques. A chaque instant, nous pouvons interroger un point de l'enviro\-nnement, et recevoir une récompense bruit\'ee. Nous nous int\'eressons d'abord au cas o\`u l'environnement est représenté par un nombre fini de points, et ensuite au cas plus g\'en\'eral o\`u l'environnement est composé d'un nombre infini de points, voire continu. Dans les deux cas, le co\^ut d'une requ\^ete pouvant \^etre \'elev\'ee, nous envisageons ainsi \`a rep\'erer au plus vite le point ($\epsilon$)-optimal.

\begin{center}
    \rule{8cm}{0.4pt}
\end{center}

\section*{\centering Abstract}


\chapter*{R\'esum\'e des travaux de thèse}
    %\addcontentsline{toc}{chapter}{R\'esum\'e des travaux de thèse (in French)}
