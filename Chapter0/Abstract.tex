\section*{\centering R\'esum\'e}

Dans cette th\`ese, nous \'etudions le probl\`eme d'optimisation s\'equentielle dans des enviro\-nnements stochastiques. A chaque instant, nous pouvons interroger un point de l'enviro\-nnement, et recevoir une récompense bruit\'ee. Nous nous concentrons d'abord sur le cas o\`u l'environnement est représenté par un nombre fini de points, et ensuite sur le cas plus g\'en\'eral o\`u l'environnement est composé d'un nombre infini d\'enombrable de points, voire continu. Dans les deux cas, le co\^ut d'une requ\^ete pouvant \^etre \'elev\'ee, nous envisageons ainsi \`a rep\'erer au plus vite le point (quasi)-optimal. Cette \'etude est motiv\'ee par de nombreux sc\'enarios r\'eels comme, entre autres, essai clinique, test A/B, optimisation des placements publicitaires. Ainsi pour terminer, nous nous int\'eressons en particulier \`a l'une de ces applications plus importantes pour la communaut\'e d'apprentissage statistique, c'est-\`a-dire l'optimisation des hyper-param\`etres.

\begin{center}
    \rule{8cm}{0.4pt}
\end{center}

\section*{\centering Abstract}

In this thesis, we study the problem of sequential optimization under stochastic environments. At each round, we can query a data point from the environment, and receive a noisy reward. We first focus on the case where the environment is abstracted as a finite search space, then we investigate also on a more general setting where the environment is composed of an infinite number of points or even continuous. In both cases, the cost of a single query would be high, and we thus aim at identify the (near)-optimum as efficiently as possible. The whole study is motivated by numerous real scenarios including, but not limited to, clinical trial, A/B testing, advertisement placement optimization. We therefore conclude by some particular focus on one of its most important contributions for the machine learning community, \emph{i.e.} hyper-parameter optimization.


\chapter*{R\'esum\'e des travaux de thèse}
    %\addcontentsline{toc}{chapter}{R\'esum\'e des travaux de thèse (in French)}
