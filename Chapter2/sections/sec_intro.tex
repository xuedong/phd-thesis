%!TEX root = ../Chapter2.tex
\section{Introduction}\label{sec:t3c.intro}

In multi-armed bandits, a learner repeatedly chooses an \emph{arm} to play, and receives a reward from the associated unknown probability distribution. When the task is \emph{best-arm} identification (BAI), the learner is not only asked to sample an arm at each stage, but is also asked to output a recommendation (i.e., a guess for the arm with the largest mean reward) after a certain period. Unlike in another well-studied bandit setting, the learner is not interested in maximizing the sum of rewards gathered during the exploration (or minimizing \emph{regret}), but only cares about the quality of her recommendation. As such, BAI is a particular \emph{pure exploration} setting~\citep{bubeck2009pure}.

Formally, we consider a finite-arm bandit model, which is a collection of $K$ probability distributions, called arms $\cA\eqdef\{1,\ldots,K\}$, parametrized by their means $\mu_1, \ldots, \mu_K$. We assume the (unknown) best arm is unique and we denote it by $I^\star \eqdef \argmax_i \mu_i$. A best-arm identification strategy $(I_n, J_n, \tau)$ consists of three components. The first is a \emph{sampling rule}, which selects an arm $I_n$ at round $n$. At each round $n$, a vector of rewards $\bY_n = (Y_{n,1},\cdots,Y_{n,K})$ is generated for all arms independently from past observations, but only $Y_{n,I_n}$ is revealed to the learner. Let $\cF_n$ be the $\sigma$-algebra generated by  $(U_1,I_1,Y_{1,I_1},\cdots,U_n,I_{n},Y_{n,I_{n}})$, then $I_n$ is $\cF_{n-1}$-measurable, i.e., it can only depend on the past $n-1$ observations, and some exogeneous randomness, materialized into $U_{n-1} \sim \cU([0,1])$. The second component is a $\cF_{n}$-measurable \emph{recommendation rule} $J_n$, which returns a guess for the best arm, and thirdly, the  \emph{stopping rule}~$\tau$, a stopping time with respect to $\left(\cF_{n}\right)_{n \in \mathbb{N}}$, decides when the exploration is over.

BAI has already been studied within several theoretical frameworks. In this paper we consider the fixed-confidence setting, introduced by \cite{even-dar2003confidence}, in which given a risk parameter $\delta$, the goal is to ensure that probability to stop and recommend a wrong arm, $\PP{J_\tau \neq I^\star}$, is smaller than $\delta$, while minimizing the expected total number of samples to make this accurate recommendation, $\EE{\tau}$. The most studied alternative is the fixed-budget setting for which the stopping rule $\tau$ is fixed to some (known) maximal budget $n$, and the goal is to minimize the error probability $\PP{J_n \neq I^\star}$ \citep{audibert2010budget}. Note that these two frameworks are very different in general and do not share transferable regret bounds (see~\citealt{carpentier2016budget} for an additional discussion).

Most of the existing sampling rules for the fixed-confidence setting depend on the risk parameter $\delta$. Some of them rely on confidence intervals such as \LUCB~\citep{kalyanakrishnan2012lucb}, \UGapE~\citep{gabillon2012ugape}, 
%\KLLUCB and \KLRacing~\citep{kaufmann2013kl}, 
or \LIL~\citep{jamieson2014lilucb}; others are based on eliminations such as \SE~\citep{even-dar2003confidence} and \EGE~\citep{karnin2013sha}. The first known sampling rule for BAI that does not depend on $\delta$ is the \emph{tracking} rule proposed by~\cite{garivier2016tracknstop}, which is proved to achieve the minimal sample complexity when combined with the Chernoff stopping rule when $\delta$ goes to zero. Such an \emph{anytime} sampling rule (neither depending on a risk $\delta$ or a budget $n$) is very appealing for applications, as advocated by \cite{jun2016atlucb}, who introduce the anytime best-arm identification framework. In this paper, we investigate another anytime sampling rule for BAI: \gls{ttts}, and propose a second anytime sampling rule: \gls{t3c}.

%\emilie{a (different type of) tracking rule was indeed proposed by Antos et al. for a different active learning problem, not sure it is work mentionning}

Thompson Sampling~\citep{thompson1933} is a Bayesian algorithm well known for regret minimization, for which it is now seen as a major competitor to \UCB-typed approaches \citep{burnetas1996optimal,auer2002ucb,cappe2013klucb}. However, it is also well known that regret minimizing algorithms cannot yield optimal performance for BAI \citep{bubeck2011pure,kaufmann2017survey} and as we opt Thompson Sampling for BAI, then its adaptation is necessary. Such an adaptation, \TTTS, was given by \citet{russo2016ttts} along with the other top-two sampling rules \TTPS and \TTVS. By choosing between two different candidate arms in each round, these sampling rules enforce the exploration of sub-optimal arms, that would be under-sampled by vanilla Thompson sampling due to its objective of maximizing rewards.

While \TTTS appears to be a good anytime sampling rule for the fixed-confidence BAI when coupled with an appropriate stopping rule, so far there is no theoretical support for this employment. Indeed, the (Bayesian-flavored) asymptotic analysis of \cite{russo2016ttts} shows that under \TTTS, the posterior probability that $I^\star$ is the best arm converges almost surely to 1 at the best possible rate. However, this property does not by itself translate into sample complexity guarantees. Since the result of \cite{russo2016ttts}, \citet{qin2017ttei} proposed and analyzed \TTEI, another Bayesian sampling rule, both in the fixed-confidence setting and in terms of posterior convergence rate. Nonetheless, similar guarantees for \TTTS have been left as an open question by \cite{russo2016ttts}. In the present paper, we answer this open question. In addition, we propose \TCC, a computationally more favorable variant of \TTTS and  extend the fixed-confidence guarantees to \TCC as well.

% and , has several desirable properties. First, it relies on choosing between two different candidates at each round to enforce the exploration and thus overcome the aforementioned drawback of Thompson sampling. Second, it does not depend on the confidence level $\delta$ just as \Track. While \TTEI has a fixed-confidence guarantee, \TTTS is only analyzed from a Bayesian asymptotic perspective (in a sense that we detail later). In the mean time one would also guess that \TTTS is a natural candidate for fixed-confidence BAI, when coupled with an appropriate stopping rule. We validate that guess in this work.

%% Too much details on anytime BAI (though well written!)

% The fact that the two previous settings provide a sampling rule that depends either on a confidence parameter $\delta$ or a budget parameter $n$ is not desirable in some real applications. To address this problem, \cite{jun2016atlucb} propose to use a \emph{doubling trick} upon fixed-budget algorithms like \SR~\citep{bubeck2009pure} and \SHA~\citep{karnin2013sha}, or use a time-varying confidence parameter when dealing with the fixed-confidence setting. This allows us to stop the learning process \emph{anytime} we want, meaning that we want the probability of $i^{\star}$ not being recommended to be decreasing as fast as possible. \TTTS, on the other hand, provides an interesting alternative that evaluates sampling rules in a Bayesian perspective, which further motivates our interest on it.
 
% \paragraph{Fixed-confidence setting}
% 
% In the fixed-confidence setting, the learner tries to obtain a fixed confidence $\delta$ about the quality of the returned arm with as few numbers of samples as possible. Thus a policy for this setting contains: (1) a \emph{sampling rule} that tells the learner which arm to sample at each stage, (2) a \emph{stopping rule} that tells the learner when to stop the learning process, and (3) a \emph{recommendation rule} that outputs a recommendation at the end of the exploration phase.

% More formally, we are given a set of $K$ arms $\cA\eqdef\{1,\ldots,K\}$ that follow $K$ unknown distributions $(\nu_k)_{1 \leq k \leq K}$, a confidence level $\delta$, and a time stopping time $\tau$ w.r.t. the observations. At each time step $t$, the learning process consists of the following actions:
% \begin{itemize}
%     \item a vector of rewards $(Y_{t,1} \sim \nu_1, \ldots, Y_{t,K} \sim \nu_K)$ is generated,
% 	\item the learner picks an arm $I_t \in \cA$ (according to the sampling rule), 
% 	\item the learner potentially recommends an arm $J_t \in \cA$ (according to the recommendation rule),
% 	\item the learner observes the reward $Y_{t,I_t}$, and
% 	\item the learner stops if $\PP{J_{\tau}\neq I^{\star}} \leq \delta$, where $I^{\star}$ is the true best arm.
% \end{itemize}
% \pmd{The last point is strange since the learner do not know this proba. the learner just decides to stop or not.}
% The goal is to obtain a small expected number of samples $\EEs{\bmu}{\tau}$, where $\bmu=(\mu_1,\mu_2,\cdots,\mu_K)$ is the underlying bandit model associated to the given set of $K$ arms.

\paragraph{Contributions} %Our main contributions are the following: 
(1) We propose a new Bayesian sampling rule, \TCC, which is inspired by \TTTS but easier to implement and computationally advantageous (2) We investigate two Bayesian stopping and recommendation rules and establish their $\delta$-correctness for a bandit model with Gaussian rewards.\footnote{hereafter `Gaussian bandits' or `Gaussian model'} (3) We provide the first sample complexity analysis of \TTTS and \TCC for a Gaussian model and our proposed stopping rule. (4) \citeauthor{russo2016ttts}'s posterior convergence results for \TTTS were obtained under restrictive assumptions on the models and priors, which exclude the two mostly used in practice: Gaussian bandits with Gaussian priors and bandits with Bernoulli rewards\footnote{hereafter `Bernoulli bandits'} with Beta priors. We prove that optimal posterior convergence rates can be obtained for those two as well.

\paragraph{Outline} In Section~\ref{sec:t3c.algorithm}, we give a reminder of \TTTS and introduce \TCC along with our proposed recommendation and stopping rules. Then, in Section~\ref{sec:related}, we describe in detail two important notions of optimality that are invoked in this paper. The main fixed-confidence analysis follows in Section~\ref{sec:t3c.confidence}, and further Bayesian optimality results are given in Section~\ref{sec:bayesian}. Numerical illustrations are given in Section~\ref{sec:t3c.experiments}.
