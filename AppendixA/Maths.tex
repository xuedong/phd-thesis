%%%%%%%%%%%%%%%%%%%%%%%%%%%%%%%%%%%%%%%%%%%%%%%%%%%%%%%%%%%%%%%%%%%%%%%%%%%%%%%%%%%%%%%%%%%%%
%%									ANNEXES 												%
%%%%%%%%%%%%%%%%%%%%%%%%%%%%%%%%%%%%%%%%%%%%%%%%%%%%%%%%%%%%%%%%%%%%%%%%%%%%%%%%%%%%%%%%%%%%%
\chapter{Mathematical Tools}\label{CHAP:MATHS}

%%%%%%%%%%%%%%%%%%%%%%%%%%%%%%%%%%%%%%%%%%%%%%%%%%%%%%%%%%%%%%%%%%%%%%%%%%%%%%%%%%%%%%%%%%%%%

\section{Reminders on Probability}\label{app:maths.proba}

$\cU([a,b])$ denotes a uniform distribution on the support $[a,b]$.

$\Beta(\cdot,\cdot)$ denotes a Beta distribution.

\subsection{One-dimensional exponential family}\label{app:maths.proba.exponential}

\begin{align}
    p_{X}(x \mid \theta ) = b(x) \exp \left[\eta (\theta ) \cdot T(x) + A(\theta )\right]\,.
\end{align}

In the whole thesis, we mainly used the following distributions from the one-dimensional exponential family.

\begin{itemize}
    \item $\Ber(\cdot)$ denotes a Bernoulli distribution.
    \item $\cB(\cdot)$ denotes a Binomial distribution.
    \item $\cN(\cdot,\cdot)$ denotes a normal distribution. Note that only normal distributions with known variance are in the one-dimensional exponential family.
\end{itemize}

%\subsection{Gaussian process}\label{app:maths.proba.gp}

\subsection{Sub-Gaussian distributions}\label{app:maths.proba.subgaussian}

\begin{definition}[sub-Gaussian]
\begin{leftbar}[defnbar]
Let $X$ be a random variable defined over a probability space $(\Omega,\mathcal{F},\Ps)$. $X$ is sub-Gaussian if there exists a constant $a\geq 0$ such that for any $\lambda\in\R$, we have
\[
    \EE{\exp(\lambda X)} \leq \exp\left\{\frac{a^2\lambda^2}{2}\right\}\,.
\]
And $X$ is called $a$-sub-Gaussian.
\end{leftbar}
\end{definition}

% \begin{example}[centered Gaussian is sub-Gaussian]
% \begin{leftbar}[examplebar]
% 	We are given a random variable $X \sim \cN(0,\sigma^2)$ with $\sigma^2$ denoting its variance. We show that $X$ is $\sigma$-sub-Gaussian. Indeed, we have $\forall \lambda\in\R$,
% 	\begin{align*}
% 		\EE{\exp(\lambda X)} &\leq \\
% 	\end{align*}
% \end{leftbar}
% \end{example}

\subsection{Martingales}\label{app:maths.proba.martingale}

\section{Concentration Inequalities}\label{app:maths.concentration}

Concentration inequalities are a omnipresent tool in \gls{mab} as they can serve as a way to bound the deviation of random variables with respect to some value (typically the expected value). In this section, we present two famous inequalities, that have been employed in this thesis.

\subsection{Hoeffding's inequality}

(Chernoff)-Hoeffding's inequality is probably the most known concentration inequality which is first studied by~\cite{hoeffding1963}. We state the Hoeffding's inequality for bounded random variables below.

\begin{theorem}\label{thm:maths.hoeffding}
\begin{leftbar}[theorembar]
    Let $X_1, X_2, \cdots, X_n$ be a sequence of independent random variables bounded in the intervals $[a_i,b_i]$ for each $i\in[n]$ respectively, then the following inequalities hold
    \[
        \PP{\bar{X}-\EE{\bar{X}} \geq t} \leq \exp\left\{-\frac{2n^2t^2}{\sum_{i=1}^n(b_i-a_i)^2}\right\}\,,
    \]
    \[
        \PP{\left|\bar{X}-\EE{\bar{X}}\right| \geq t} \leq 2\exp\left\{-\frac{2n^2t^2}{\sum_{i=1}^n(b_i-a_i)^2}\right\}\,,
    \]
    where $\bar(X)$ denotes the empirical mean of those random variables,
    \[
        \bar{X} \eqdef \frac{1}{n}(X_1+X_2+\cdots+X_n)\,.
    \]
\end{leftbar}
\end{theorem}

A generalization of the previous inequalities to sub-Gaussian random variables also exists. We do not provide details in this thesis, interested readers can refer to~\cite{vershynin2018}.

\subsection{Azuma's inequality}

Another important inequality that has been used in this thesis is Azuma-(Hoeffding)'s inequality~\citep{azuma1967}. The vanilla form of Azuma's inequality is stated as follow.

\begin{theorem}\label{thm:maths.azuma}
\begin{leftbar}[theorembar]
    Let $(X_0, X_1, X_2, \cdots)$ be a sequence of random variables, and we assume that it forms a martingale (or a super-martingale). Suppose that for any $i\in\NN$,
    \[
        |X_i-X_{i-1}|\leq c_i \text{a.s.}\,,
    \]
    with $(c_i)_{i\in\NN}$ a sequence of constants. Then for any positive integer $n$ and any real $\epsilon$, the following inequality holds,
    \[
        \PP{X_n-X_0 \geq \epsilon} \leq \exp\left\{\frac{-\epsilon^2}{2\sum_{i=1}^n c_i^2}\right\}\,.
    \]
    Symmetrically, if the sequence form a sub-martingale, then the following inequality holds,
    \[
        \PP{X_n-X_0 \leq -\epsilon} \leq \exp\left\{\frac{-\epsilon^2}{2\sum_{i=1}^n c_i^2}\right\}\,.
    \]
\end{leftbar}
\end{theorem}

The two parts of Theorem~\ref{thm:maths.azuma} can be combined together using a union bound to obtain the following two-side bound.

\begin{corollary}\label{cor:maths.azuma}
\begin{leftbar}[corollarybar]
    Let $(X_0, X_1, X_2, \cdots)$ be a martingale such that for any $i\in\NN$,
    \[
        |X_i-X_{i-1}|\leq c_i \text{a.s.}\,.
    \]
    Then we have
    \[
        \PP{X_n-X_0 \leq -\epsilon} \leq \exp\left\{\frac{-\epsilon^2}{2\sum_{i=1}^n c_i^2}\right\}\,.
    \]
\end{leftbar}
\end{corollary}

%\subsection{Bernstein's inequality}

%\section{Reproducing Kernel Hilbert Space}\label{app:maths.rkhs}

\section{Information Theory}\label{app:maths.information}

In this section, we briefly recall some fundamental notions and results of information theory that are unceasingly used in the technical proofs of this thesis. Readers can refer to~\cite{cover2006} for more details.

\subsection{Entropy}\label{app:maths.information.entropy}

Given a random variable $X: \Omega \rightarrow \cX$, the \gls{entropy} $X$ measures its uncertainty, and also defines the ultimate data compression. When the random variable is discrete, its entropy $H(X)$ is defined as follow.

\begin{definition}[entropy]\label{def:entropy}
\begin{leftbar}[defnbar]
    Let $X$ be a discrete random variable defined over a probability space $(\Omega,\mathcal{F},\Ps)$ to an arbitrary space $\cX$, with probability mass function $p_X$, then its entropy $H(X)$ is defined by
    \[
        H(X) \eqdef - \sum_{x\in\cX} p_X(x)\log p_X(x)\,.
    \]
\end{leftbar}
\end{definition}

The previous definition can be extended to continuous random variables, namely \gls{differential entropy}.

\begin{definition}[differential entropy]
\begin{leftbar}[defnbar]
    Let $X$ be a continuous random variable defined over a probability space $(\Omega,\mathcal{F},\Ps)$, with probability density function $f$, then its differential entropy $h(X)$ is defined by
    \[
        h(X) \eqdef - \int f(x)\log f(x) dx\,.
    \]
\end{leftbar}
\end{definition}

\subsection{Kullback-Leibler divergence}\label{app:maths.information.kl}

 important concept is the \textit{relative entropy} or \textit{Kullback-Leibler divergence} (KL divergence), which measures the difference between two probability distributions.

\begin{remark}
\begin{leftbar}[remarkbar]
	KL divergence is not a distance since it does not satisfy the symmetry property in an usual distance definition.
\end{leftbar}
\end{remark}

Before properly defining the KL divergence, let us first give the definition of an important prerequisite notion of \textit{absolutely continuous} probability measures.

\begin{definition}[absolutely continuous probability measures]\label{def:maths.absolute_continuous}
\begin{leftbar}[defnbar]
	Let $\mathbb{P}$ and $\mathbb{Q}$ be two probability measures defined on a measurable space $(\Omega,\mathcal{F})$. If for any event $F \in \mathcal{F}$ such that $\mathbb{Q}(F) = 0$, we have also $\mathbb{P}(F) = 0$, then one says that $\mathbb{P}$ is absolutely continuous w.r.t $\mathbb{Q}$, and is denoted as $\mathbb{P} \ll \mathbb{Q}$.
\end{leftbar}
\end{definition}

The KL divergence is then defined as follow.

\begin{definition}[KL divergence]\label{def:maths.kl}
\begin{leftbar}[defnbar]
	For two probability measures $\mathbb{P}$ and $\mathbb{Q}$, if $\mathbb{P} \ll \mathbb{Q}$, then the KL divergence is defined as
	\[
		KL(\mathbb{P} \lVert \mathbb{Q}) \eqdef \int \log(\frac{d\mathbb{P}}{d\mathbb{Q}})d\mathbb{P}.
	\]
\end{leftbar}
\end{definition}

A very important property of KL divergence is its non-negativity, which is established by Gibbs' theory.

\begin{theorem}[Gibbs' inequality]\label{thm:maths.gibbs}
\begin{leftbar}[theorembar]
	For two probability measures $\mathbb{P}$ and $\mathbb{Q}$, if $\mathbb{P} \ll \mathbb{Q}$, then $KL(\mathbb{P} \lVert \mathbb{Q}) \geq 0$. The equality holds if and only if $\mathbb{P} = \mathbb{Q}$ $\mathbb{P}$-almost everywhere.
\end{leftbar}
\end{theorem}

% \begin{proof}
% 	We prove the result in the discrete case. Suppose $P = \{p_1, p_2, \ldots, p_n\}$ and $Q = \{q_1, q_2, \ldots, q_n\}$ two probability distributions.

% 	We know that $\log(\cdot)$ is a concave function, thus $-\log(\cdot)$ is a convex function. Using Jensen's inequality, we have
% 	\begin{align*}
% 		D(\mathbb{P} \lVert \mathbb{Q}) & = \sum_{i=1}^n p_i \log(p_i) \frac{\log(p_i)}{\log(q_i)} \\
% 										& = - \sum_{i=1}^n p_i \log(p_i) \frac{\log(q_i)}{\log(p_i)} \\
% 										& \geq - \log (\sum_{i=1}^n p_i \frac{q_i}{p_i}) \\
% 										& = 0.
% 	\end{align*}
% \end{proof}

\subsection{Two special cases: Gaussian and Bernoulli}\label{app:maths.information.examples}

For probability distributions in the one-dimensional exponential family, we can simply represent the KL-divergence by their means. For example, if $\mu_1$ and $\mu_2$ are respectively the means of $\mathbb{P}$ and $\mathbb{Q}$, then we can write
\[
    KL(\mathbb{P};\mathbb{Q}) = d(\mu_1;\mu_2)\,.
\]

For some particular probability distributions, simple closed-form expressions can be deduced. In Example~\ref{ex:maths.kl_gaussian} and Example~\ref{ex:maths.kl_bernoulli}, we show the expressions for Gaussian and Bernoulli distributions.

\begin{example}[KL-divergence between two Gaussian distributions]\label{ex:maths.kl_gaussian}
\begin{leftbar}[examplebar]
	We compute the KL divergence between two normal distributions $\mathbb{P}\sim\cN(\mu_1,\sigma_1)$ and $\mathbb{Q}\sim\cN(\mu_2,\sigma_2)$.
	\[
		KL(\mathbb{P} \lVert \mathbb{Q}) = \log \frac{\sigma_2}{\sigma_1} + \frac{\sigma_1^2+(\mu_1-\mu_2)^2}{2\sigma_2^2} - \frac{1}{2}.
	\]
In particular, if $\sigma\eqdef\sigma_1=\sigma_2$, then the two distributions are parameterized by their means, we can thus denote by
	\[
		d(\mu_1;\mu_2) \eqdef KL(\mathbb{P} \lVert \mathbb{Q}) = \frac{(\mu_1-\mu_2)^2}{2\sigma^2}.
	\]
\end{leftbar}
\end{example}

\begin{example}[KL divergence between two Bernoulli distributions]\label{ex:maths.kl_bernoulli}
\begin{leftbar}[examplebar]
    We compute the KL divergence between two Bernoulli distributions $\mathbb{P}\sim\Ber(\mu_1)$ and $\mathbb{Q}\sim\Ber(\mu_2)$, we denote by
     \[
        kl(\mu_1;\mu_2) \eqdef KL(\mathbb{P} \lVert \mathbb{Q}) =  \mu_1 \ln \left( \frac{\mu_1}{\mu_2} \right) + (1-\mu_1) \ln  \left( \frac{1-\mu_1}{1-\mu_2} \right).
    \]
\end{leftbar}
\end{example}
