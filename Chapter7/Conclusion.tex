%%%%%%%%%%%%%%%%%%%%%%%%%%%%%%%%%%%%%%%%%%%%%%%%%%%%%%%%%%%%%%%%%%%%%%%%%%%%%%%%%%%%%%%%%%%%%
%%									Chapitre 7											%
%%%%%%%%%%%%%%%%%%%%%%%%%%%%%%%%%%%%%%%%%%%%%%%%%%%%%%%%%%%%%%%%%%%%%%%%%%%%%%%%%%%%%%%%%%%%%
\chapter{General Conclusion and Perspectives}\label{CHAP:CONCLUSION}
	\citationChap{
	\begin{CJK*}{UTF8}{gbsn}
	将来现在将来,与现在有意义,才与将来会有意义。
    \end{CJK*}
	}{\begin{CJK*}{UTF8}{gbsn}
	鲁迅
    \end{CJK*}}
	\minitoc
	\newpage

%%%%%%%%%%%%%%%%%%%%%%%%%%%%%%%%%%%%%%%%%%%%%%%%%%%%%%%%%%%%%%%%%%%%%%%%%%%%%%%%%%%%%%%%%%%%%

\section{General Discussion} 

In this thesis, we studied the multi-armed bandit problem in an optimization fashion. In particular, we investigated three different settings of best-arm identification (in a broad sense) in the first three chapters. 

We first studied \gls{bai} in its simplest formulation (Chapter~\ref{CHAP:T3C}), that is bandits with scalar payoffs. We treated the problem with some Bayesian machinery and answered to one open question raised by~\cite{russo2016ttts} on the sample complexity. By showing the ($\beta$)-asymptotic optimality of \gls{ttts} and providing a computationally faster alternative \gls{t3c}, we further advocated the use of Bayesian algorithms for \gls{bai}.

In the next chapter (Chapter~\ref{CHAP:LGC}), we studied the linear setting with the hope of extending previous Bayesian algorithms while keeping the same sample-complexity guarantee. We argued that previous notion of complexities for linear bandits \gls{bai} did not allow us to achieve the asymptotic optimality. Although the result was not satisfying regarding the Bayesian extensions, we managed to propose an alternative \gls{lg} using a saddle-point approach that is asymptotically optimal whilst remaining computationally-friendly.

The third part (Chapter~\ref{CHAP:GPO}) consists of a rather different setting where we aimed to optimize a target function over a continuous-armed space with minimum regularity assumptions. We were interested in designing algorithms that are adaptive to the smoothness. Taking inspiration from \gls{poo}, we proposed a new general cross-validation scheme \gls{gpo}. Compared to \gls{poo} that is only able to encapsulate hierarchical-bandit algorithms with a cumulative-regret guarantee, \gls{gpo} is able to encapsulate algorithms with simple-regret guarantees.

The first chapters mostly came up with strongly theoretically grounded algorithms in the context of different sequential optimization settings, while in Chapter~\ref{CHAP:DTTTS} we also explored a more practical topic, namely hyper-parameter optimization. Existing methods often require to fix an \emph{ad hoc} number of configurations to test, while we managed to propose a dynamic algorithm \gls{dttts} that do not need such a workaround. It is worth noting that \gls{dttts} can also simply serve as a heuristic for \gls{infinitely-armed bandits}, but without any theoretical guarantees. Analysis of \gls{dttts} appears to be difficult due to its dynamic nature, and is left for future work.

\section{Future Perspectives}

\paragraph{Follow-ups of the previous research.} 
A prominent follow-up is to further investigate whether both theoretically and practically efficient Bayesian algorithms exist for linear \gls{bai} (or even more general structure). As discussed in Chapter~\ref{CHAP:LGC}, our first attempts to extend \gls{ttts} to the linear setting leads to a dead end from a theoretical point of view, but still shows some promising experimental performance. I think it is worth putting some efforts on the topic as Bayesian methods could probably avoid resolving complicated optimization problems that is required in most of the current existing state-of-the-art algorithms.

On the other hand, as advocated for example by~\cite{locatelli2016thresholding}, the fixed-budget and fixed-confidence settings are drastically different. In the future, I would like to put more focus on the fixed-budget setting, and ideally propose finite-time analysis for \gls{ttts}.

\paragraph{Further investigation on hyper-parameter tuning.}
Even if the motivation of this thesis was to design efficient \gls{hpo} algorithms, the present thesis does not explore the topic very deeply. A main reason is that I find it hard to propose new \emph{task-agnostic} algorithms that improve over existing methods significantly. Indeed, the most important contribution of \Hyperband{} is to introduce a more clever evaluation mode that allocates more resources to more promising configurations. This trick has given way to a considerable improvement over previous methods, in particular on tasks using large \gls{dl} models. Since \Hyperband{}, however, with a lot of recent work trying to advance the state-of-the-art (see Section~\ref{sec:dttts.survey}), no real breakthrough has been achieved in the field.

In my opinion, the future (or even ongoing) trend of the domain would be to focus on designing more efficient task-specific algorithms. Specific methods could eventually achieve better performance on specific tasks than more general approaches. To some extent, the recent progress on \gls{nas} is one running example to support this intuition as \gls{rl} methods have been successfully applied on \gls{nas} (see e.g.~\citealt{zoph2017}).

A future direction that attracts me a lot is thus to explore the potential of \gls{hpo} techniques in different industrial applications. For example, one important business need of companies dealing with microchips is the compiler optimization. The methodology will thus be a bit different: we need to design algorithms based on the task itself rather than proposing a general algorithm and testing it on different tasks.

On the other hand, sample-efficient \gls{hpo} remains an active research field since there is not only neural networks that need automated hyper-parameter tuning. Indeed, in many data science challenges with real-world datasets, classical machine learning methods (in particular ensemble methods like \XG{}) often dominate. Bandit-inspired \gls{hpo} methods can still be plausible candidates for those tasks and I am very interested in discovering further in this direction.

\paragraph{Link to reinforcement learning.}
Beyond \gls{mab}, I would also like to work more on \gls{rl} in the future, whether it is about the theoretical foundation or finding real applications of \gls{rl}. \gls{rl} is a richer and more challenging domain than \gls{mab}, yet has strong links with \gls{mab}, in particular linear bandits. Indeed, \gls{rl} extends upon contextual bandits by allowing for long-term consequences. More precisely, for contextual (linear) bandits, the actions only affect the current reward, whereas for \gls{rl}, they can also affect the future rewards through the evolution of the context.

%It is worth elaborating a bit more on the link between linear bandits (or more general contextual bandits) and \gls{rl} in addition to the discussion in Section~\ref{sec:intro.context.rl}. 

One possible topic that I would like to investigate -- a topic that is also highly related to \gls{bai} -- is the problem of \gls{bpi} in \gls{mdp}s. Similar to \gls{bai}, the goal of \gls{bpi} is to devise learning algorithms that are able to return the best policy as early as possible. \cite{marjani2020bpi} adapt \Track to \gls{bpi} in discounted \gls{mdp}s, with the help of a generative model. It is interesting to see if our algorithms for linear bandits \gls{bai} can help design sound \gls{bpi} algorithms in a more general context (without generative model).

In the long term, I will be interested in filling the gap between \gls{rl} theory and practical usable \gls{rl}. For example, a large number of recent theoretical \gls{rl} work has adopted a linear function approximation assumption first studied by~\cite{jin2019linear} to various \gls{rl} problem settings. This assumption is interesting from a theoretical point of view, but remains quite unrealistic in practice. It would be interesting to find a more realistic assumption while keeping (or improving) the current guarantees.

\paragraph{Societal impact.}
Last but not least, there is an emerging attention that has been paid on the societal impact recently. I value also a lot these societal factors or constraints that should be leveraged in the current machine learning research. I am particularly interested in safety which is often required in many real business projects (see e.g.~\citealt{garcelon2020attack,leurent2020robust}), and also fairness that plays an crucial role in empowering the inclusion.

Some of my recently finished or on-going projects focus on some of those aspects (in particular, safety for linear bandits~\citealt{shang2021safe} and a novel bandit setting designed upon fairness considerations~\citealt{shang2020vector}) and it is something that I definitely want to take into account in my future work.
