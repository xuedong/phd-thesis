% !TEX root = ../Chapter4.tex
\section{Detailed regret analysis for HCT under Assumption~\ref{ass1}}\label{app:gpo.hct}

\subsection{Preliminaries}\label{proof:pre}
We first fix some constants and introduce some additional notation that are needed for the proof of Theorem~\ref{thm:hct}.
\begin{itemize}
    \item $c_1 \triangleq (\rho/(3\nu))^{1/8}$, $c \triangleq 2\sqrt{1/(1-\rho)}$
    \item $\forall 1\leq h\leq H(t)$ and $t>0$, $\mathcal{I}_h(t)$ denotes the set of all nodes created by \HCT at level $h$ up to step $t$
    \item $\forall 1\leq h\leq H(t)$ and $t>0$, $\mathcal{I}_h^{+}(t)$ denotes the subset of $\mathcal{I}_h(t)$ which contains only the internal nodes
    \item At each step $t$, $(h_t,i_t)$ denotes the node selected by the algorithm.
    \item $\mathcal{C}_{h,i} \triangleq \{t = 1,\cdots,n : (h_t,i_t) = (h,i)\}$
    \item $\mathcal{C}_{h,i}^{+} \triangleq \underset{j\in\{0,\ldots,K-1\}}{\bigcup} \cC_{h+1,Ki-j}$
    \item $\overline{t}_{h,i} \triangleq \max_{t\in\mathcal{C}_{h,i}} t$ denotes the last time $(h,i)$ has been selected
    \item $\tilde{t}_{h,i} \triangleq \max_{t\in\mathcal{C}_{h,i}^{+}} t$ denotes the last time when one of its children has been selected
    \item $t_{h,i} \triangleq \min\{t: T_{h,i}(t) \geq \tau_h(t)\}$ is the time when $(h,i)$ is expanded
    \item For any $t$, let $y_t\triangleq(r_t,x_t)$ be a random variable, we define the filtration $\cF_t$ as a $\sigma$-algebra generated by $(y_1,\ldots,y_t)$.
\end{itemize}
Another important notion in  \HCT  is the threshold $\tau_h$ on the number of pulls needed before a node at level $h$ can be expanded.
The threshold $\tau_h$ is chosen such that the two confidence terms in $U_{h,i}$ are roughly equivalent, that is,
\[
\nu\rho^h \simeq c\sqrt{\frac{\operatorname{log}(1/\tilde{\delta}(t^+))}{\tau_h(t)}}\cdot
\]
Therefore, we choose
\[
\tau_h(t) \triangleq \ceil{\frac{c^2\operatorname{log}(1/\tilde{\delta}(t^+))}{\nu^2}\rho^{-2h}}\!.
\]
Since $t^+$ is defined as $2^{\ceil{\operatorname{log}(t)}}$, we have $t \leq t^+ \leq 2t$. In addition, $\log$ is an increasing function, thus we have
\begin{equation} \label{eq1}
    \frac{c^2}{\nu^2}\rho^{-2h} \leq \frac{c^2 \operatorname{log}(1/\tilde{\delta}(t))}{\nu^2}\rho^{-2h} \leq \tau_h(t) \leq \frac{c^2\operatorname{log}(2/\tilde{\delta}(t))}{\nu^2}\rho^{-2h},
\end{equation}
where the first inequality follows from the fact that $0<\tilde{\delta}(t)\leq1/2$. We begin our analysis by bounding the maximum depth of the trees constructed by \HCT.

\subsection{Maximum depth of the tree (proof of Lemma~\ref{lemma_depth})}\label{proof:lemma_depth}
%\setcounter{scratchcounter}{\value{theorem}}\setcounter{theorem}{\the\numexpr\getrefnumber{lemma_depth}-1}
\restalemmadepth*
%\setcounter{theorem}{\the\numexpr\value{scratchcounter}}

\begin{proof}
The deepest tree that can be constructed by \HCT is a linear one, where at each level one unique node is expanded. In such case,   $|\mathcal{I}_h^{+}(n)|=1$ and $|\mathcal{I}_h(n)|=K$ for all $h<H(n)$. Therefore, we have
\begin{align*}
    n & = \sum_{h=0}^{H(n)}\sum_{i\in\mathcal{I}_h(n)} T_{h,i}(n) \\
      & \geq \sum_{h=0}^{H(n)-1}\sum_{i\in\mathcal{I}_h^{+}(n)} T_{h,i}(n) \\
      & \geq \sum_{h=0}^{H(n)-1}\sum_{i\in\mathcal{I}_h^{+}(n)} T_{h,i}(t_{h,i}) \\
      & \geq \sum_{h=0}^{H(n)-1}\sum_{i\in\mathcal{I}_h^{+}(n)} \tau_h(t_{h,i}) && \text{definition of } t_{h,i} \\
      & \geq \sum_{h=0}^{H(n)-1}\frac{c^2}{\nu^2}\rho^{-2h} && \text{ineq.\,(\ref{eq1})} \\
      & \geq \frac{(c\rho)^2}{\nu^2}\rho^{-2H(n)}H(n) && \text{since } h\leq H(n)-1 \\
      & \geq \frac{(c\rho)^2}{\nu^2}\rho^{-2H(n)}.
\end{align*}
By solving this expression, we obtain
\begin{align*}
    H(n) & \leq \frac{1}{2}\operatorname{log}\left(                \frac{n\nu^2}{c^2\rho^2} \right)/\operatorname{log}(1/\rho) \\
         & \leq \frac{1}{2(1-\rho)}\operatorname{log}\left(
    \frac{n\nu^2}{c^2\rho^2} \right) && \text{follows from } \operatorname{log}(1/\rho)\geq 1-\rho \\
	& \leq \ceil{\frac{1}{2(1-\rho)}\operatorname{log}\left(
    \frac{n\nu^2}{c^2\rho^2} \right)} \\
         & \eqdef H_{\max}(n).
\end{align*}
\end{proof}

\subsection{High-probability event}\label{proof:lemma_event}

In Section~\ref{sec:gpo.analysis}, we described the favorable event $\xi_t$.  We now define it precisely. We first define a set $\cL_t$ that contains all possible nodes in trees of maximum depth $H_{\max}(t),$
\[
\mathcal{L}_t \triangleq \bigcup\limits_{\mathcal{T}:\operatorname{depth}(\mathcal{T})\leq H_{\max}(t)} \operatorname{Nodes}(\mathcal{T})
\]
and we recall the definition of the favorable event
\[
\xi_t \triangleq \left\{ \forall (h,i)\in\mathcal{L}_t,  |\hat{\mu}_{h,i}(t) - \mu_{h,i}| \leq c\sqrt{\frac{\operatorname{log}(1/\tilde{\delta}(t))}{T_{h,i}(t)}} \right\}\!\cdot\]
Next, we prove that our favorable event holds with high probability.

\begin{lemma} \label{lemma_event}
With $c_1$ and $c$ defined in Section~\ref{proof:pre}, for any fixed round~$t$,
\[
\mathbb{P}\left[ \xi_t \right] \geq 1-\frac{4\delta}{3t^6}\cdot
\]
\end{lemma}

\begin{proof}
Letting $\hat{\mu}_{h,i,s}$ denote the empirical mean reward of the first $s$ noisy evaluations of $f$ in $x_{h,i}$, we upper-bound the probability of the complementary event $\xi_t^c$ as

\begin{align*}
    \mathbb{P}\left[ \xi_t^c \right]
    & \leq \sum_{(h,i)\in\mathcal{L}_t}\sum_{s=1}^{t} \mathbb{P}\left[ |\hat{\mu}_{h,i,s} - \mu_{h,i}| \geq c\sqrt{\frac{\operatorname{log}(1/\tilde{\delta}(t))}{s}} \right] && \text{union bound} \\
    & \leq \sum_{(h,i)\in\mathcal{L}_t}\sum_{s=1}^{t} 2\operatorname{exp}\left( -2c^2\operatorname{log}(1/\tilde{\delta}(t)) \right) && \text{Chernoff-Hoeffding inequality} \\
    & = 2\operatorname{exp}\left( -2c^2\operatorname{log}(1/\tilde{\delta}(t)) \right) t|\mathcal{L}_t| \\
    & = 2(\tilde{\delta}(t))^{2c^2}t|\mathcal{L}_t| \\
    & \leq 2(\tilde{\delta}(t))^{2c^2}t 2^{H_{max}(t)+1} \\
    & = 2(\tilde{\delta}(t))^{2c^2}t 2^{\ceil{\frac{1}{2(1-\rho)}\operatorname{log}\left(\frac{n\nu^2}{c^2\rho^2} \right)}+1} && \text{Lemma~\ref{lemma_depth}} \\
    & \leq 8t(\tilde{\delta}(t))^{2c^2} \left( \frac{t\nu^2}{c^2\rho^2} \right)^{\frac{1}{2(1-\rho)}} \\
    & \leq 8t\left(\frac{\delta}{t} (\rho/(3\nu))^{1/8})^{\frac{8}{1-\rho}}\right) \left( \frac{t\nu^2(1-\rho)}{4\rho^2}\right)^{\frac{1}{2(1-\rho)}} && \text{plugging in values of $c$ and $c_1$} \\
    & = 8t\left(\frac{\delta}{t}\right)^{\frac{8}{1-\rho}} \left(\frac{\rho}{3\nu}\right)^{\frac{1}{1-\rho}} t^{\frac{1}{2(1-\rho)}} \left( \frac{\nu\sqrt{1-\rho}}{2\rho}\right)^{\frac{1}{1-\rho}} \\
    & \leq \frac{4}{3} \delta t^{\frac{-2\rho-13}{2(1-\rho)}} \\
    & \leq \frac{4\delta}{3t^6}\cdot
\end{align*}

\end{proof}

\subsection{Failing confidence bound}\label{proof:lemma_failing}

We decompose the regret of \HCT into two terms depending on whether $\xi_t$ holds. Let us define $\Delta_t \eqdef f^\star - r_t$. Then, we decompose the regret as
\[
R_n^{\HCT} = \sum_{t=1}^n \Delta_t = \sum_{t=1}^n \Delta_t \bOne_{\xi_t} + \sum_{t=1}^n \Delta_t \bOne_{\xi_t^c} = R_n^{\xi} + R_n^{\xi^c}.
\]
The failing confidence term $R_n^{\xi^c}$ is bounded by the following lemma.
\begin{lemma} \label{lemma_failing}
With $c_1$ and $c$ defined in Section~\ref{proof:pre}, when the favorable event does not hold, the regret of {\HCT} is with probability $1-\delta/(5n^2)$ bounded as
\[
R_n^{\xi^c} \leq \sqrt{n}.
\]
\end{lemma}

\begin{proof}
We split the term into rounds from $1$ to $\sqrt{n}$ and the rest,
\[
R_n^{\xi^c} = \sum_{t=1}^n \Delta_t \bOne_{\xi_t^c} = \sum_{t=1}^{\sqrt{n}} \Delta_t \bOne_{\xi_t^c} + \sum_{t=\sqrt{n}+1}^n \Delta_t \bOne_{\xi_t^c}.
\]
The first term can be bounded trivially by $\sqrt{n}$ since $|\Delta_t|\leq 1$. Next, we show that the probability that the second term is non zero is bounded by $\delta/(5n^2)$.
\begin{align*}
    \mathbb{P}\left[ \sum_{t=\sqrt{n}+1}^n \Delta_t \bOne_{\xi_t^c} > 0 \right] & = \mathbb{P}\left[ \bigcup\limits_{t=\sqrt{n}+1}^n \xi_t^c \right]          \\
            & \leq \sum_{t=\sqrt{n}+1}^n \mathbb{P}\left[ \xi_t^c \right] && \text{union bound} \\
            & \leq \sum_{t=\sqrt{n}+1}^n \frac{\delta}{t^6} && \text{Lemma~\ref{lemma_event}} \\
            & \leq \int_{\sqrt{n}}^{\infty} \frac{\delta}{t^6} \, \mathrm{d}t \\
            & = \frac{\delta}{5n^{5/2}} \\
            & \leq \frac{\delta}{5n^2}\cdot
\end{align*}

\end{proof}

\subsection{Proof of Theorem~\ref{thm:hct}}\label{proof:thm}
\restathm*
\noindent
%Now we come back to Theorem~\ref{thm:hct}.
For the sake of simplicity, we denote $d(\nu,C,\rho)$ as $d$ in the rest of this section. We study the regret under events $\{\xi_t\}_t$ and prove that
\[
R_n^{\HCT(\nu,\rho)} \leq 2\sqrt{2n\operatorname{log}(\frac{4n^2}{\delta}}) + 3\left(\frac{2^{3d+7}\nu^d KC\rho^d}{(1-\rho)^{2}}\right)^{\frac{1}{d+2}}\left(\operatorname{log}\left(\frac{2n}{\delta}\sqrt[8]{\frac{3\nu}{\rho}}\right)\right)^{\frac{1}{d+2}}n^{\frac{d+1}{d+2}}
\]
holds with probability $1-\delta$. We decompose the proof into 3 steps.

\paragraph{Step 1: Decomposition of the regret.}
We start by further decomposing the instantaneous regret into two terms,
\[
\Delta_t = f^\star - r_t = f^\star - f(x_{h_t,i_t}) + f(x_{h_t,i_t}) - r_t = \Delta_{h_t,i_t} + \hat{\Delta}_t.
\]
The regret of \HCT when confidence intervals hold can thus be rewritten as
\begin{equation} \label{eq2}
R_n^{\xi} = \sum_{t=1}^n \Delta_{h_t,i_t} \bOne_{\xi_t} + \sum_{t=1}^n \hat{\Delta}_t \bOne_{\xi_t} \leq \sum_{t=1}^n \Delta_{h_t,i_t} \bOne_{\xi_t} + \sum_{t=1}^n \hat{\Delta}_t = \tilde{R}_n^{\xi} + \hat{R}_n^{\xi}.
\end{equation}
We notice that the sequence $\{\hat{\Delta}_t\}_{t=1}^n$ is a bounded martingale difference sequence since $\mathbb{E}\left[\hat{\Delta}_t|\mathcal{F}_{t-1}\right]=0$ and $|\hat{\Delta}_t|\leq 1$. Thus, we apply the Azuma's inequality on this sequence and obtain
\begin{equation} \label{eq3}
\hat{R}_n^{\xi} \leq \sqrt{2n\operatorname{log}\left(\frac{4n^2}{\delta}\right)}
\end{equation}
with probability $1-\delta/(4n^2)$.


\paragraph{Step 2: Preliminary bound on the regret of selected nodes and their parents.}
Now we proceed with the bound of the first term $\tilde{R}_n^{\xi}$.
Recall that $P_t$ is the optimistic path traversed by \HCT at round $t$.
Let $(h',i')\in P_t$ and $(h^{''},i^{''})$ be the node which immediately follows $(h',i')$ in $P_t$. By definition of $B$-values and $U$-values, we have
\begin{equation} \label{eq4}
B_{h',i'}(t) \leq \underset{j\in\{0,\ldots,K-1\}}{\max} \left\{B_{h'+1,Ki'-j}(t)\right\} = B_{h^{''},i^{''}}(t),
\end{equation}
where the last equality follows from the fact that the subroutine \texttt{OptTraverse} selects the node with the largest $B$-value. By iterating the previous inequality along  the path $P_t$ until the selected node $(h_t,i_t)$ and its parent $(h_t^p,i_t^p)$, we obtain
\[
\forall (h',i')\in P_t, B_{h',i'}(t) \leq B_{h_t,i_t}(t) \leq U_{h_t,i_t}(t),
\]
\[
\forall (h',i')\in P_t\setminus \left\{(h_t,i_t)\right\}, B_{h',i'}(t) \leq B_{h_t^p,i_t^p}(t) \leq U_{h_t^p,i_t^p}(t).
\]
Since the root, which is an optimal node, is in $P_t$, there exists at least one optimal node $(h^\star,i^\star)$ in path $P_t$. As a result, we have
\begin{align}
    B_{h^\star,i^\star}(t) & \leq U_{h_t,i_t}(t), \label{eq5} \\
    B_{h^\star,i^\star}(t) & \leq U_{h_t^p,i_t^p}(t). \label{eq6}
\end{align}
We now expand (\ref{eq5}) on both sides under $\xi_t$. First, we have
\begin{align}
    U_{h_t,i_t}(t) & \eqdef \hat{\mu}_{h_t,i_t}(t) + \nu\rho^{h_t} +     c\sqrt{\frac{\operatorname{log}(1/\tilde{\delta}(t^+))}{T_{h_t,i_t}(t)}} %\\
                   %&
                   \leq f(x_{h_t,i_t}) + \nu\rho^{h_t} + 2c\sqrt{\frac{\operatorname{log}(1/\tilde{\delta}(t^+))}{T_{h_t,i_t}(t)}} \label{eq8}
\end{align}
and the same holds for the parent of the selected node,
\begin{equation*} %\label{eq9}
U_{h_t^p,i_t^p}(t) \leq f(x_{h_t^p,i_t^p}) +\nu\rho^{h_t^p} + 2c\sqrt{\frac{\operatorname{log}(1/\tilde{\delta}(t^+))}{T_{h_t^p,i_t^p}(t)}}\cdot
\end{equation*}
By Lemma~\ref{upper}, we know that $U_{h^\star,i^\star}(t)$ is a valid upper bound on $f^\star$. If an optimal node $(h^\star,i^\star)$ is a leaf, then $B_{h^\star,i^\star}(t)=U_{h^\star,i^\star}(t)$ is also a valid upper bound on $f^\star$. Otherwise, there always exists a leaf which contains the maximum for which $(h^\star,i^\star)$ is its ancestor. Now, if we propagate the bound backward from this leaf to $(h^\star,i^\star)$ through (\ref{eq4}), we have that $B_{h^\star,i^\star}(t)$ is still a valid upper bound on $f^\star$. Thus for any optimal node $(h^\star,i^\star)$, at round~$t$ under $\xi_t$, we have
\begin{equation} \label{eq10}
B_{h^\star,i^\star}(t) \geq f^\star.
\end{equation}
We combine (\ref{eq10}) with (\ref{eq5}) and (\ref{eq8}) to obtain
\begin{equation*} %\label{eq11}
\Delta_{h_t,i_t} \eqdef f^\star - f(x_{h_t,i_t}) \leq \nu\rho^{h_t} + 2c\sqrt{\frac{\operatorname{log}(1/\tilde{\delta}(t^+))}{T_{h_t,i_t}(t)}}\cdot
\end{equation*}
The same result holds for its parent,
\begin{equation*} %\label{eq12}
\Delta_{h_t^p,i_t^p} \eqdef f^\star - f(x_{h_t^p,i_t^p}) \leq \nu\rho^{h_t^p} + 2c\sqrt{\frac{\operatorname{log}(1/\tilde{\delta}(t^+))}{T_{h_t^p,i_t^p}(t)}}\cdot
\end{equation*}
We now refine the two above expressions. The subroutine \texttt{OptTraverse} tells us that \HCT only selects a node when $T_{h,i}(t)<\tau_h(t)$. Therefore, by definition of $\tau_{h_t}(t)$, we have
\begin{equation} \label{eq13}
\Delta_{h_t,i_t} \leq 3c\sqrt{\frac{\operatorname{log}(2/\tilde{\delta}(t))}{T_{h_t,i_t}(t)}}\cdot
\end{equation}
On the other hand,  \texttt{OptTraverse} tells us that $T_{h_t^p,i_t^p}(t)\geq\tau_{h_t^p}(t)$, thus
\begin{equation*} %\label{eq14}
\Delta_{h_t^p,i_t^p} \leq 3\nu\rho^{h_t^p},
\end{equation*}
which means that every selected node has a parent which is $(3\nu\rho^{h_t-1})$-optimal.


\paragraph{Step 3: Bound on the cumulative regret.}
We return to  term $\tilde{R}_n^{\xi}$ and split it into different depths. Let $1\leq \bar{H} \leq H(n)$ be a constant that we  fix later. We have
\begin{align*}
    \tilde{R}_n^{\xi} &\eqdef \sum_{t=1}^n \Delta_{h_t,i_t} \bOne_{\xi_t} \\
                              &\leq \sum_{h=0}^{H(n)}\sum_{i\in\mathcal{I}_h(n)}\sum_{t=1}^n \Delta_{h,i} \bOne_{(h_t,i_t)=(h,i)} \bOne_{\xi_t} \\
                              &\leq \sum_{h=0}^{H(n)}\sum_{i\in\mathcal{I}_h(n)}\sum_{t=1}^n 3c\sqrt{\frac{\operatorname{log}(2/\tilde{\delta}(t))}{T_{h,i}(t)}} \bOne_{(h_t,i_t)=(h,i)} && \text{ineq.\,(\ref{eq13})} \\
                              &= \sum_{h=0}^{\bar{H}}\sum_{i\in\mathcal{I}_h(n)}\sum_{t=1}^n 3c\sqrt{\frac{\operatorname{log}(2/\tilde{\delta}(t))}{T_{h,i}(t)}} \bOne_{(h_t,i_t)=(h,i)} + \sum_{h=\bar{H}+1}^{H(n)}\sum_{i\in\mathcal{I}_h(n)}\sum_{t=1}^n 3c\sqrt{\frac{\operatorname{log}(2/\tilde{\delta}(t))}{T_{h,i}(t)}} \bOne_{(h_t,i_t)=(h,i)} \\
                              &\leq \sum_{h=0}^{\bar{H}}\sum_{i\in\mathcal{I}_h(n)}\sum_{s=1}^{\tau_h(\bar{t}_{h,i})} 3c\sqrt{\frac{\operatorname{log}(2/\tilde{\delta}(\bar{t}_{h,i}))}{s}} + \sum_{h=\bar{H}+1}^{H(n)}\sum_{i\in\mathcal{I}_h(n)}\sum_{s=1}^{T_{h,i}(n)} 3c\sqrt{\frac{\operatorname{log}(2/\tilde{\delta}(\bar{t}_{h,i}))}{s}} \\
                              &\leq \sum_{h=0}^{\bar{H}}\sum_{i\in\mathcal{I}_h(n)}\int_{1}^{\tau_h(\bar{t}_{h,i})} 3c\sqrt{\frac{\operatorname{log}(2/\tilde{\delta}(\bar{t}_{h,i}))}{s}} \, \mathrm{d}s + \sum_{h=\bar{H}+1}^{H(n)}\sum_{i\in\mathcal{I}_h(n)}\int_{1}^{T_{h,i}(n)} 3c\sqrt{\frac{\operatorname{log}(2/\tilde{\delta}(\bar{t}_{h,i}))}{s}} \, \mathrm{d}s \\
                              &\leq \sum_{h=0}^{\bar{H}}\sum_{i\in\mathcal{I}_h(n)} 6c\sqrt{\tau_h(\bar{t}_{h,i})\operatorname{log}(2/\tilde{\delta}(\bar{t}_{h,i}))} + \sum_{h=\bar{H}+1}^{H(n)}\sum_{i\in\mathcal{I}_h(n)} 6c\sqrt{T_{h,i}(n)\operatorname{log}(2/\tilde{\delta}(\bar{t}_{h,i}))} \\
                              &= 6c\left(\underbrace{\sum_{h=0}^{\bar{H}}\sum_{i\in\mathcal{I}_h(n)} \sqrt{\tau_h(\bar{t}_{h,i})\operatorname{log}(2/\tilde{\delta}(\bar{t}_{h,i}))}}_\text{(a)} + \underbrace{\sum_{h=\bar{H}+1}^{H(n)}\sum_{i\in\mathcal{I}_h(n)} \sqrt{T_{h,i}(n)\operatorname{log}(2/\tilde{\delta}(\bar{t}_{h,i}))}}_\text{(b)}\right).
\end{align*}
We bound separately the terms (a) and (b). Since $\bar{t}_{h,i}\leq n$, we have
\begin{equation*} %\label{eq15}
\text{(a)} \leq \sum_{h=0}^{\bar{H}}\sum_{i\in\mathcal{I}_h(n)} \sqrt{\tau_h(n)\operatorname{log}(2/\tilde{\delta}(n))} \leq \sum_{h=0}^{\bar{H}}|\mathcal{I}_h(n)| \sqrt{\tau_h(n)\operatorname{log}(2/\tilde{\delta}(n))}.
\end{equation*}
Notice that the covering tree is $K$-ary and therefore $|\mathcal{I}_h(n)| \leq K|\mathcal{I}_{h-1}(n)|$. Recall that \HCT only selects a node $(h_t,i_t)$ when its parent is $3\nu\rho^{h_t-1}$-optimal. Therefore, by definition of the near-optimality dimension,
\begin{equation*} %\label{eq16}
|\mathcal{I}_h(n)| \leq| K\mathcal{I}_{h-1}(n)| \leq KC\rho^{-d(h-1)},
\end{equation*}
where $d$ is the near-optimality dimension. As a result, for term (a), we obtain that
\begin{align*}
    \text{(a)} & \leq \sum_{h=0}^{\bar{H}} KC\rho^{-d(h-1)} \sqrt{\tau_h(n)\log(2/\tilde{\delta}(n))} \\
               & = \sum_{h=0}^{\bar{H}} KC\rho^{-d(h-1)} \sqrt{\frac{c^2\log(2/\tilde{\delta}(n))}{\nu^2}\rho^{-2h}\log(2/\tilde{\delta}(n))} && \text{ineq.\,(\ref{eq1})} \\
               & = KC\rho^d \frac{c\operatorname{log}(2/\tilde{\delta}(n))}{\nu} \sum_{h=0}^{\bar{H}} \rho^{-h(d+1)}.
\end{align*}
Consequently, we  bound (a) as
\begin{equation} \label{eq17}
\text{(a)} \leq KC\rho^d \frac{c\operatorname{log}\left(2/\tilde{\delta}(n)\right)}{\nu} \frac{\rho^{-\bar{H}(d+1)}}{1-\rho}\cdot
\end{equation}
We proceed to bound the second term (b). By the Cauchy-Schwarz inequality,
\begin{equation*} %\label{eq18}
\text{(b)} \leq \sqrt{\sum_{h=\bar{H}+1}^{H(n)}\sum_{i\in\mathcal{I}_h(n)} \operatorname{log}\left(2/\tilde{\delta}\left(\bar{t}_{h,i}\right)\right)} \sqrt{\sum_{h=\bar{H}+1}^{H(n)}\sum_{i\in\mathcal{I}_h(n)} T_{h,i}(n)} \leq \sqrt{n\sum_{h=\bar{H}+1}^{H(n)}\sum_{i\in\mathcal{I}_h(n)} \operatorname{log}\left(2/\tilde{\delta}\left(\bar{t}_{h,i}\right)\right)},
\end{equation*}
where we trivially bound the second square-root factor by the total number of pulls. Now consider the first square-root factor. Recall that the \HCT algorithm only selects a node when $T_{h,i}(t)\geq\tau_h(t)$ for its parent. We therefore have $T_{h,i}(\tilde{t}_{h,i})\geq\tau_h(\tilde{t}_{h,i})$ and the following sequence of inequalities,
%\vfil\newpage
\begin{align*}
    n & = \sum_{h=0}^{H(n)}\sum_{i\in\mathcal{I}_h(n)} T_{h,i}(n) \\
      &\geq \sum_{h=0}^{H(n)-1}\sum_{i\in\mathcal{I}_h^+(n)} T_{h,i}(n) \\
      & \geq \sum_{h=0}^{H(n)-1}\sum_{i\in\mathcal{I}_h^+(n)} T_{h,i}(\tilde{t}_{h,i}) && \text{$\tilde{t}_{h,i}$ well defined for $i\in\mathcal{I}_h^+(n)$} \\
      & \geq \sum_{h=0}^{H(n)-1}\sum_{i\in\mathcal{I}_h^+(n)} \tau_h(\tilde{t}_{h,i}) \\
      & \geq \sum_{h=\bar{H}}^{H(n)-1}\sum_{i\in\mathcal{I}_h^+(n)} \tau_h(\tilde{t}_{h,i}) \\
      & = \sum_{h=\bar{H}}^{H(n)-1}\sum_{i\in\mathcal{I}_h^+(n)} \frac{c^2\operatorname{log}(1/\tilde{\delta}(\tilde{t}_{h,i}^+)))}{\nu^2}\rho^{-2h}
         \end{align*}
      \begin{align*}
       \sum_{h=\bar{H}}^{H(n)-1}&\sum_{i\in\mathcal{I}_h^+(n)} \frac{c^2\operatorname{log}(1/\tilde{\delta}(\tilde{t}_{h,i}^+)))}{\nu^2}\rho^{-2h}  \\
       & \hspace{-0.8cm}  \geq \sum_{h=\bar{H}}^{H(n)-1}\sum_{i\in\mathcal{I}_h^+(n)} \frac{c^2\operatorname{log}(1/\tilde{\delta}(\tilde{t}_{h,i}^+)))}{\nu^2}\rho^{-2\bar{H}} \\
      & \hspace{-0.8cm}= \frac{c^2\rho^{-2\bar{H}}}{\nu^2} \sum_{h=\bar{H}}^{H(n)-1}\sum_{i\in\mathcal{I}_h^+(n)} \operatorname{log}(1/\tilde{\delta}(\tilde{t}_{h,i}^+))) \\
      & \hspace{-0.8cm}= \frac{c^2\rho^{-2\bar{H}}}{\nu^2} \sum_{h=\bar{H}}^{H(n)-1}\sum_{i\in\mathcal{I}_h^+(n)} \operatorname{log}(1/\tilde{\delta}(\left[\max(\bar{t}_{h+1,2i-1},\bar{t}_{h+1,2i})\right]^+)) && \text{since $\tilde{t}_{h,i}=\max(\bar{t}_{h+1,2i-1},\bar{t}_{h+1,2i})$} \\
      & \hspace{-0.8cm}= \frac{c^2\rho^{-2\bar{H}}}{\nu^2} \sum_{h=\bar{H}}^{H(n)-1}\sum_{i\in\mathcal{I}_h^+(n)} \operatorname{log}(1/\tilde{\delta}(\max(\bar{t}_{h+1,2i-1}^+,\bar{t}_{h+1,2i}^+))) && \text{$\forall t_1,t_2,\left[\max(t_1,t_2)\right]^+=\max(t_1^+,t_2^+)$} \\
      & \hspace{-0.8cm}= \frac{c^2\rho^{-2\bar{H}}}{\nu^2} \sum_{h=\bar{H}}^{H(n)-1}\sum_{i\in\mathcal{I}_h^+(n)} \max(\operatorname{log}(1/\tilde{\delta}(\bar{t}_{h+1,2i-1}^+)),\operatorname{log}(1/\tilde{\delta}(\bar{t}_{h+1,2i}^+))) \\
      & \geq \frac{c^2\rho^{-2\bar{H}}}{\nu^2} \sum_{h=\bar{H}}^{H(n)-1}\sum_{i\in\mathcal{I}_h^+(n)} \frac{\operatorname{log}(1/\tilde{\delta}(\bar{t}_{h+1,2i-1}^+))+\operatorname{log}(1/\tilde{\delta}(\bar{t}_{h+1,2i}^+))}{2} \\
      &\hspace{-0.8cm} = \frac{c^2\rho^{-2\bar{H}}}{\nu^2} \sum_{h=\bar{H}+1}^{H(n)}\sum_{i\in\mathcal{I}_{h-1}^+(n)} \frac{\operatorname{log}(1/\tilde{\delta}(\bar{t}_{h,2i-1}^+))+\operatorname{log}(1/\tilde{\delta}(\bar{t}_{h,2i}^+))}{2} && \text{change of variables} \\
      &\hspace{-0.8cm} = \frac{c^2\rho^{-2\bar{H}}}{2\nu^2} \sum_{h=\bar{H}+1}^{H(n)}\sum_{i\in\mathcal{I}_h^+(n)} \operatorname{log}(1/\tilde{\delta}(\bar{t}_{h,i}^+)).
\end{align*}
In the above derivation, the last equality relies on the fact that for any $h>0$, $\mathcal{I}_h^+(n)$ covers all the internal nodes at level $h$ and therefore its children cover $\mathcal{I}_{h+1}(n)$. We  thus obtain
\begin{equation} \label{eq19}
\sum_{h=\bar{H}+1}^{H(n)}\sum_{i\in\mathcal{I}_h^+(n)} \operatorname{log}(1/\tilde{\delta}(\bar{t}_{h,i}^+)) \leq \frac{2\nu^2\rho^{2\bar{H}}n}{c^2}\cdot
\end{equation}
On the other hand, we have
\begin{align*}
    \text{(b)} &\leq \sqrt{n\sum_{h=\bar{H}+1}^{H(n)}\sum_{i\in\mathcal{I}_h(n)} \operatorname{log}(2/\tilde{\delta}(\bar{t}_{h,i}))}
               \leq \sqrt{n\sum_{h=\bar{H}+1}^{H(n)}\sum_{i\in\mathcal{I}_h(n)} 2\operatorname{log}(1/\tilde{\delta}(\bar{t}_{h,i}))} \\
               &\leq \sqrt{n\sum_{h=\bar{H}+1}^{H(n)}\sum_{i\in\mathcal{I}_h(n)} 2\operatorname{log}(1/\tilde{\delta}(\bar{t}_{h,i}^+))}, && \text{since $\bar{t}_{h,i}\leq\bar{t}_{h,i}^+$.}
\end{align*}
By plugging (\ref{eq19}) into above expression, we get
\begin{equation} \label{eq20}
\text{(b)} \leq \frac{2\nu\rho^{\bar{H}}n}{c}\cdot
\end{equation}
Now if we combine  (\ref{eq20}) with (\ref{eq17}), we  bound  $\tilde{R}_n^{\xi}$ as
\begin{equation} \label{eq21}
\tilde{R}_n^{\xi} \leq 6\nu\left[KC\rho^d \frac{c^2\log(2/\tilde{\delta}(n))}{\nu^2} \frac{\rho^{-\bar{H}(d+1)}}{1-\rho} + 2\rho^{\bar{H}}n\right]\!\cdot
\end{equation}
We  choose $\bar{H}$ to minimize the above bound by equalizing the two terms in the sum and  obtain
\begin{equation}
\rho^{\bar{H}} = \left( \frac{KC\rho^dc^2\log(2/\tilde{\delta}(n))}{2n(1-\rho)\nu^2} \right)^{\frac{1}{d+2}}\!\!\!\!\!\!\!\!,
\end{equation}
which after being plugged into (\ref{eq21}) gives
\begin{equation} \label{eq23}
\tilde{R}_n^{\xi} \leq 24\nu \left( \frac{KC\rho^dc^2\log(2/\tilde{\delta}(n))}{2(1-\rho)\nu^2} \right)^{\frac{1}{d+2}}n^{\frac{d+1}{d+2}}.
\end{equation}
Finally, combining (\ref{eq23}), (\ref{eq3}), and Lemma~\ref{lemma_failing}, we obtain
\begin{align*}
    R_n^{\HCT} & \leq \sqrt{n} + \sqrt{2n\operatorname{log}(\frac{4n^2}{\delta}}) + 24\nu\left(\frac{2KC\rho^d}{(1-\rho)^2\nu^2}\right)^{\frac{1}{d+2}}\left(\operatorname{log}\left(\frac{2n}{\delta}\sqrt[8]{\frac{3\nu}{\rho}}\right)\right)^{\frac{1}{d+2}}n^{\frac{d+1}{d+2}} \\
                              & = \sqrt{n} + \sqrt{2n\operatorname{log}(\frac{4n^2}{\delta}}) + 3\left(\frac{2^{3d+7}\nu^d KC\rho^d}{(1-\rho)^{2}}\right)^{\frac{1}{d+2}}\left(\operatorname{log}\left(\frac{2n}{\delta}\sqrt[8]{\frac{3\nu}{\rho}}\right)\right)^{\frac{1}{d+2}}n^{\frac{d+1}{d+2}} \\
                              & \leq 2\sqrt{2n\operatorname{log}(\frac{4n^2}{\delta}}) + 3\left(\frac{2^{3d+7}\nu^d KC\rho^d}{(1-\rho)^{2}}\right)^{\frac{1}{d+2}}\left(\operatorname{log}\left(\frac{2n}{\delta}\sqrt[8]{\frac{3\nu}{\rho}}\right)\right)^{\frac{1}{d+2}}n^{\frac{d+1}{d+2}}
\end{align*}
with probability $1-\delta$.
