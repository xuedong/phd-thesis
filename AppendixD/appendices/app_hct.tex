% !TEX root = ../Chapter5.tex
\section{Detailed regret analysis for \HCT{} under Assumption~\ref{ass:gpo.local}}\label{app:gpo.hct}

We begin our analysis by bounding the maximum depth of the trees constructed by \gls{hct}.

\subsection{Maximum depth of the tree (proof of Lemma~\ref{lemma:gpo.depth})}\label{proof:lemma_depth}
%\setcounter{scratchcounter}{\value{theorem}}\setcounter{theorem}{\the\numexpr\getrefnumber{lemma_depth}-1}
\restalemmadepth*
%\setcounter{theorem}{\the\numexpr\value{scratchcounter}}

\begin{proof}
The deepest tree that can be constructed by \gls{hct} is a linear one, where at each level one unique node is expanded. In such case,   $|\mathcal{I}_h^{+}(N)|=1$ and $|\mathcal{I}_h(N)|=K$ for all $h<H(N)$. Therefore, we have
\begingroup
\allowdisplaybreaks
\begin{align}
    N & = \sum_{h=0}^{H(N)}\sum_{i\in\mathcal{I}_h(N)} T_{h,i}(N) \nonumber \\
      & \geq \sum_{h=0}^{H(N)-1}\sum_{i\in\mathcal{I}_h^{+}(N)} T_{h,i}(N) \nonumber \\
      & \geq \sum_{h=0}^{H(N)-1}\sum_{i\in\mathcal{I}_h^{+}(N)} T_{h,i}(n_{h,i}) \nonumber \\
      & \geq \sum_{h=0}^{H(N)-1}\sum_{i\in\mathcal{I}_h^{+}(N)} \tau_h(n_{h,i}) \label{eq:gpo.depth1} \\
      & \geq \sum_{h=0}^{H(N)-1}\frac{c^2}{\nu^2}\rho^{-2h} \label{eq:gpo.depth2} \\
      & \geq \frac{(c\rho)^2}{\nu^2}\rho^{-2H(N)}H(N) \label{eq:gpo.depth3} \\
      & \geq \frac{(c\rho)^2}{\nu^2}\rho^{-2H(N)} \nonumber \,.
\end{align}
\endgroup
In the previous reasoning, \eqref{eq:gpo.depth1} is based on the definition of $n_{h,i}$, \eqref{eq:gpo.depth2} is due to inequality~\eqref{eq:gpo.pre}, and~\eqref{eq:gpo.depth3} holds since $h\leq H(N)-1$.

By solving this expression, we obtain
\begingroup
\allowdisplaybreaks
\begin{align}
    H(N) & \leq \frac{1}{2}\log\left(\frac{N\nu^2}{c^2\rho^2} \right)/\log(1/\rho) \nonumber \\
         & \leq \frac{1}{2(1-\rho)}\log\left(
    \frac{N\nu^2}{c^2\rho^2} \right) \label{eq:gpo.depth4} \\
	& \leq \ceil{\frac{1}{2(1-\rho)}\log\left(
    \frac{N\nu^2}{c^2\rho^2} \right)} \nonumber \\
         & \eqdef H_{\max}(N) \nonumber \,.
\end{align}
\endgroup
where \eqref{eq:gpo.depth4} follows from $\log(1/\rho)\geq 1-\rho$.
\end{proof}

\subsection{High-probability event}\label{proof:lemma_event}

In Section~\ref{sec:gpo.analysis}, we described the favorable event $\xi_n$.  We now define it precisely. We first define a set $\cL_n$ that contains all possible nodes in trees of maximum depth $H_{\max}(n),$
\[
\mathcal{L}_n \triangleq \bigcup\limits_{\mathcal{T}:\operatorname{depth}(\mathcal{T})\leq H_{\max}(n)} \operatorname{Nodes}(\mathcal{T})
\]
and we recall the definition of the favorable event
\[
\xi_n \triangleq \left\{ \forall (h,i)\in\mathcal{L}_n,  |\hat{\mu}_{h,i}(n) - \mu_{h,i}| \leq c\sqrt{\frac{\log(1/\tilde{\delta}(n))}{T_{h,i}(n)}} \right\}\!\cdot\]
Next, we prove that our favorable event holds with high probability.

\begin{lemma}\label{lemma:gpo.event}
\begin{leftbar}[lemmabar]
With $c_1$ and $c$ defined in Section~\ref{sec:gpo.pre}, for any fixed round~$n$,
\[
    \mathbb{P}\left[ \xi_n \right] \geq 1-\frac{4\delta}{3n^6}\,.
\]
\end{leftbar}
\end{lemma}

\begin{proof}
Let $\hat{\mu}_{h,i,s}$ be the empirical mean reward of the first $s$ noisy evaluations of $f$ in $x_{h,i}$, we upper-bound the probability of the complementary event $\xi_n^c$ as

\begingroup
\allowdisplaybreaks
\begin{align}
    \mathbb{P}\left[ \xi_n^c \right]
    & \leq \sum_{(h,i)\in\mathcal{L}_n}\sum_{s=1}^{n} \mathbb{P}\left[ |\hat{\mu}_{h,i,s} - \mu_{h,i}| \geq c\sqrt{\frac{\log(1/\tilde{\delta}(n))}{s}} \right] \label{eq:gpo.event1} \\
    & \leq \sum_{(h,i)\in\mathcal{L}_n}\sum_{s=1}^{n} 2\operatorname{exp}\left( -2c^2\log(1/\tilde{\delta}(n)) \right) \label{eq:gpo.event2} \\
    & = 2\operatorname{exp}\left( -2c^2\log(1/\tilde{\delta}(n)) \right) n|\mathcal{L}_n| \nonumber \\
    & = 2(\tilde{\delta}(n))^{2c^2}n|\mathcal{L}_n| \nonumber \\
    & \leq 2(\tilde{\delta}(n))^{2c^2}n 2^{H_{max}(n)+1} \nonumber \\
    & = 2(\tilde{\delta}(n))^{2c^2}n 2^{\ceil{\frac{1}{2(1-\rho)}\log\left(\frac{n\nu^2}{c^2\rho^2} \right)}+1} \label{eq:gpo.event3} \\
    & \leq 8n(\tilde{\delta}(n))^{2c^2} \left( \frac{t\nu^2}{c^2\rho^2} \right)^{\frac{1}{2(1-\rho)}} \nonumber \\
    & \leq 8n\left(\frac{\delta}{n} (\rho/(3\nu))^{1/8})^{\frac{8}{1-\rho}}\right) \left( \frac{n\nu^2(1-\rho)}{4\rho^2}\right)^{\frac{1}{2(1-\rho)}} \label{eq:gpo.event4} \\
    & = 8n\left(\frac{\delta}{n}\right)^{\frac{8}{1-\rho}} \left(\frac{\rho}{3\nu}\right)^{\frac{1}{1-\rho}} n^{\frac{1}{2(1-\rho)}} \left( \frac{\nu\sqrt{1-\rho}}{2\rho}\right)^{\frac{1}{1-\rho}} \nonumber \\
    & \leq \frac{4}{3} \delta n^{\frac{-2\rho-13}{2(1-\rho)}} \nonumber \\
    & \leq \frac{4\delta}{3n^6} \nonumber \,.
\end{align}
\endgroup
Here, \eqref{eq:gpo.event1} is derived using a union bound, and~\eqref{eq:gpo.event2} is due to the Hoeffding inequality (see Appendix~\ref{app:maths.concentration} for details). \eqref{eq:gpo.event3} is a result of Lemma~\ref{lemma:gpo.depth} and finally~\eqref{eq:gpo.event4} is obtained by plugging in values of $c$ and $c_1$.
\end{proof}

\subsection{Failing confidence bound}\label{proof:lemma_failing}

We decompose the regret of \gls{hct} into two terms depending on whether $\xi_n$ holds. Let us define $\Delta_n \eqdef f^\star - r_n$. Then, we decompose the regret as
\[
    R_N^{\gls{hct}} = \sum_{n=1}^N \Delta_n = \sum_{n=1}^N \Delta_n \bOne_{\xi_n} + \sum_{n=1}^N \Delta_n \bOne_{\xi_n^c} = R_N^{\xi} + R_N^{\xi^c}\,.
\]
The failing confidence term $R_N^{\xi^c}$ is bounded by the following lemma.
\begin{lemma}\label{lemma:gpo.failing}
\begin{leftbar}[lemmabar]
With $c_1$ and $c$ defined in Section~\ref{app:gpo.notation}, when the favorable event does not hold, the regret of {\gls{hct}} is with probability $1-\delta/(5N^2)$ bounded as
\[
    R_N^{\xi^c} \leq \sqrt{N}\,.
\]
\end{leftbar}
\end{lemma}

\begin{proof}
We split the term into rounds from $1$ to $\sqrt{N}$ and the rest,
\[
R_N^{\xi^c} = \sum_{n=1}^N \Delta_n \bOne_{\xi_n^c} = \sum_{n=1}^{\sqrt{N}} \Delta_n \bOne_{\xi_n^c} + \sum_{n=\sqrt{N}+1}^N \Delta_n \bOne_{\xi_n^c}.
\]
The first term can be bounded trivially by $\sqrt{N}$ since $|\Delta_n|\leq 1$. Next, we show that the probability that the second term is non zero is bounded by $\delta/(5N^2)$.
\begingroup
\allowdisplaybreaks
\begin{align}
    \mathbb{P}\left[ \sum_{n=\sqrt{N}+1}^N \Delta_n \bOne_{\xi_n^c} > 0 \right] & = \mathbb{P}\left[ \bigcup\limits_{n=\sqrt{N}+1}^N \xi_n^c \right] \nonumber \\
            & \leq \sum_{n=\sqrt{N}+1}^N \mathbb{P}\left[ \xi_n^c \right] \label{eq:gpo.failing1} \\
            & \leq \sum_{n=\sqrt{N}+1}^N \frac{\delta}{n^6} \label{eq:gpo.failing2} \\
            & \leq \int_{\sqrt{N}}^{\infty} \frac{\delta}{n^6} \, \mathrm{d}n \nonumber \\
            & = \frac{\delta}{5N^{5/2}} \nonumber \\
            & \leq \frac{\delta}{5N^2} \nonumber \,.
\end{align}
\endgroup
In the previous reasoning, \eqref{eq:gpo.failing1} is achieved again by a simple union bound, and~\eqref{eq:gpo.failing2} can be obtained by Lemma~\ref{lemma:gpo.event}
\end{proof}

\subsection{Proof of Theorem~\ref{thm:gpo.hct}}\label{proof:thm}
\restathmhct*
\noindent
%Now we come back to Theorem~\ref{thm:hct}.
For the sake of simplicity, we denote $d(\nu,C,\rho)$ as $d$ in the rest of this section. We study the regret under events $\{\xi_n\}_n$ and prove that
\[
R_N^{\gls{hct}(\nu,\rho)} \leq 2\sqrt{2N\log(\frac{4N^2}{\delta}}) + 3\left(\frac{2^{3d+7}\nu^d KC\rho^d}{(1-\rho)^{2}}\right)^{\frac{1}{d+2}}\left(\log\left(\frac{2N}{\delta}\sqrt[8]{\frac{3\nu}{\rho}}\right)\right)^{\frac{1}{d+2}}N^{\frac{d+1}{d+2}}
\]
holds with probability $1-\delta$. We decompose the proof into 3 steps.

\paragraph{Step 1: Decomposition of the regret.}
We start by further decomposing the instantaneous regret into two terms,
\[
\Delta_n = f^\star - r_n = f^\star - f(x_{h_n,i_n}) + f(x_{h_n,i_n}) - r_n = \Delta_{h_n,i_n} + \hat{\Delta}_n.
\]
The regret of \gls{hct} when confidence intervals hold can thus be rewritten as
\begin{equation} \label{eq2}
R_N^{\xi} = \sum_{n=1}^N \Delta_{h_n,i_n} \bOne_{\xi_n} + \sum_{n=1}^N \hat{\Delta}_n \bOne_{\xi_n} \leq \sum_{n=1}^N \Delta_{h_n,i_n} \bOne_{\xi_n} + \sum_{n=1}^N \hat{\Delta}_n = \tilde{R}_N^{\xi} + \hat{R}_N^{\xi}.
\end{equation}
We notice that the sequence $\{\hat{\Delta}_n\}_{n=1}^N$ is a bounded martingale difference sequence since $\mathbb{E}\left[\hat{\Delta}_n|\mathcal{F}_{n-1}\right]=0$ and $|\hat{\Delta}_n|\leq 1$. Thus, we apply the Azuma's inequality on this sequence and obtain
\begin{equation} \label{eq3}
\hat{R}_N^{\xi} \leq \sqrt{2N\log\left(\frac{4N^2}{\delta}\right)}
\end{equation}
with probability $1-\delta/(4N^2)$.


\paragraph{Step 2: Preliminary bound on the regret of selected nodes and their parents.}
Now we proceed with the bound of the first term $\tilde{R}_N^{\xi}$.
Recall that $P_n$ is the optimistic path traversed by \gls{hct} at round $t$.
Let $(h',i')\in P_n$ and $(h^{''},i^{''})$ be the node which immediately follows $(h',i')$ in $P_n$. By definition of $B$-values and $U$-values, we have
\begin{equation} \label{eq4}
B_{h',i'}(n) \leq \underset{j\in\{0,\ldots,K-1\}}{\max} \left\{B_{h'+1,Ki'-j}(n)\right\} = B_{h^{''},i^{''}}(n),
\end{equation}
where the last equality follows from the fact that the subroutine \texttt{OptTraverse} selects the node with the largest $B$-value. By iterating the previous inequality along  the path $P_n$ until the selected node $(h_n,i_n)$ and its parent $(h_n^p,i_n^p)$, we obtain
\[
\forall (h',i')\in P_n, B_{h',i'}(n) \leq B_{h_n,i_n}(n) \leq U_{h_n,i_n}(n),
\]
\[
\forall (h',i')\in P_n\setminus \left\{(h_n,i_n)\right\}, B_{h',i'}(n) \leq B_{h_n^p,i_n^p}(n) \leq U_{h_n^p,i_n^p}(n).
\]
Since the root, which is an optimal node, is in $P_n$, there exists at least one optimal node $(h^\star,i^\star)$ in path $P_n$. As a result, we have
\begin{align}
    B_{h^\star,i^\star}(n) & \leq U_{h_n,i_n}(n), \label{eq:gpo.bvalue1} \\
    B_{h^\star,i^\star}(n) & \leq U_{h_n^p,i_n^p}(n). \label{eq:gpo.bvalue2}
\end{align}
We now expand (\ref{eq:gpo.bvalue1}) on both sides under $\xi_n$. First, we have
\begin{align}
    U_{h_n,i_n}(n) & \eqdef \hat{\mu}_{h_n,i_n}(n) + \nu\rho^{h_n} +     c\sqrt{\frac{\log(1/\tilde{\delta}(n^+))}{T_{h_n,i_n}(n)}} \nonumber \\
                   &\leq f(x_{h_n,i_n}) + \nu\rho^{h_n} + 2c\sqrt{\frac{\log(1/\tilde{\delta}(n^+))}{T_{h_n,i_n}(n)}} \label{eq:gpo.ucb}
\end{align}
and the same holds for the parent of the selected node,
\begin{equation*}
    U_{h_n^p,i_n^p}(n) \leq f(x_{h_n^p,i_n^p}) +\nu\rho^{h_n^p} + 2c\sqrt{\frac{\log(1/\tilde{\delta}(n^+))}{T_{h_n^p,i_n^p}(n)}}\,.
\end{equation*}
By Lemma~\ref{lemma:gpo.upper}, we know that $U_{h^\star,i^\star}(n)$ is a valid upper bound on $f^\star$. If an optimal node $(h^\star,i^\star)$ is a leaf, then $B_{h^\star,i^\star}(n)=U_{h^\star,i^\star}(n)$ is also a valid upper bound on $f^\star$. Otherwise, there always exists a leaf which contains the maximum for which $(h^\star,i^\star)$ is its ancestor. Now, if we propagate the bound backward from this leaf to $(h^\star,i^\star)$ through (\ref{eq4}), we have that $B_{h^\star,i^\star}(n)$ is still a valid upper bound on $f^\star$. Thus for any optimal node $(h^\star,i^\star)$, at round~$n$ under $\xi_n$, we have
\begin{equation} \label{eq10}
B_{h^\star,i^\star}(n) \geq f^\star.
\end{equation}
We combine (\ref{eq10}) with (\ref{eq:gpo.bvalue1}) and (\ref{eq:gpo.ucb}) to obtain
\begin{equation*} %\label{eq11}
\Delta_{h_n,i_n} \eqdef f^\star - f(x_{h_n,i_n}) \leq \nu\rho^{h_n} + 2c\sqrt{\frac{\log(1/\tilde{\delta}(n^+))}{T_{h_n,i_n}(n)}}\cdot
\end{equation*}
The same result holds for its parent,
\begin{equation*} %\label{eq12}
\Delta_{h_n^p,i_n^p} \eqdef f^\star - f(x_{h_n^p,i_n^p}) \leq \nu\rho^{h_n^p} + 2c\sqrt{\frac{\log(1/\tilde{\delta}(n^+))}{T_{h_n^p,i_n^p}(n)}}\cdot
\end{equation*}
We now refine the two above expressions. The subroutine \texttt{OptTraverse} tells us that \gls{hct} only selects a node when $T_{h,i}(N)<\tau_h(n)$. Therefore, by definition of $\tau_{h_n}(n)$, we have
\begin{equation}\label{eq:gpo.delta}
    \Delta_{h_n,i_n} \leq 3c\sqrt{\frac{\log(2/\tilde{\delta}(n))}{T_{h_n,i_n}(n)}}\cdot
\end{equation}
On the other hand,  \texttt{OptTraverse} tells us that $T_{h_n^p,i_n^p}(n)\geq\tau_{h_n^p}(n)$, thus
\begin{equation*} %\label{eq14}
\Delta_{h_n^p,i_n^p} \leq 3\nu\rho^{h_n^p},
\end{equation*}
which means that every selected node has a parent which is $(3\nu\rho^{h_n-1})$-optimal.


\paragraph{Step 3: Bound on the cumulative regret.}
We return to  term $\tilde{R}_N^{\xi}$ and split it into different depths. Let $1\leq \bar{H} \leq H(N)$ be a constant that we  fix later. We have
\begingroup
\allowdisplaybreaks
\begin{align}
    \tilde{R}_N^{\xi} &\eqdef \sum_{n=1}^N \Delta_{h_n,i_n} \bOne_{\xi_n} \nonumber \\
                              &\leq \sum_{h=0}^{H(N)}\sum_{i\in\mathcal{I}_h(N)}\sum_{n=1}^N \Delta_{h,i} \bOne_{(h_n,i_n)=(h,i)} \bOne_{\xi_n} \nonumber \\
                              &\leq \sum_{h=0}^{H(N)}\sum_{i\in\mathcal{I}_h(N)}\sum_{n=1}^N 3c\sqrt{\frac{\log(2/\tilde{\delta}(n))}{T_{h,i}(n)}} \bOne_{(h_n,i_n)=(h,i)} \label{eq:gpo.cumulative1} \\
                              &= \sum_{h=0}^{\bar{H}}\sum_{i\in\mathcal{I}_h(N)}\sum_{n=1}^N 3c\sqrt{\frac{\log(2/\tilde{\delta}(n))}{T_{h,i}(n)}} \bOne_{(h_n,i_n)=(h,i)} \nonumber \\
                              &+ \sum_{h=\bar{H}+1}^{H(N)}\sum_{i\in\mathcal{I}_h(N)}\sum_{n=1}^N 3c\sqrt{\frac{\log(2/\tilde{\delta}(n))}{T_{h,i}(n)}} \bOne_{(h_n,i_n)=(h,i)} \nonumber \\
                              &\leq \sum_{h=0}^{\bar{H}}\sum_{i\in\mathcal{I}_h(N)}\sum_{s=1}^{\tau_h(\bar{n}_{h,i})} 3c\sqrt{\frac{\log(2/\tilde{\delta}(\bar{n}_{h,i}))}{s}} + \sum_{h=\bar{H}+1}^{H(N)}\sum_{i\in\mathcal{I}_h(N)}\sum_{s=1}^{T_{h,i}(N)} 3c\sqrt{\frac{\log(2/\tilde{\delta}(\bar{n}_{h,i}))}{s}} \nonumber \\
                              &\leq \sum_{h=0}^{\bar{H}}\sum_{i\in\mathcal{I}_h(N)}\int_{1}^{\tau_h(\bar{n}_{h,i})} 3c\sqrt{\frac{\log(2/\tilde{\delta}(\bar{n}_{h,i}))}{s}} \, \mathrm{d}s \nonumber \\
                              &+ \sum_{h=\bar{H}+1}^{H(N)}\sum_{i\in\mathcal{I}_h(N)}\int_{1}^{T_{h,i}(N)} 3c\sqrt{\frac{\log(2/\tilde{\delta}(\bar{n}_{h,i}))}{s}} \, \mathrm{d}s \nonumber \\
                              &\leq \sum_{h=0}^{\bar{H}}\sum_{i\in\mathcal{I}_h(N)} 6c\sqrt{\tau_h(\bar{n}_{h,i})\log(2/\tilde{\delta}(\bar{n}_{h,i}))} + \sum_{h=\bar{H}+1}^{H(N)}\sum_{i\in\mathcal{I}_h(N)} 6c\sqrt{T_{h,i}(N)\log(2/\tilde{\delta}(\bar{n}_{h,i}))} \nonumber \\
                              &= 6c\left(\underbrace{\sum_{h=0}^{\bar{H}}\sum_{i\in\mathcal{I}_h(N)} \sqrt{\tau_h(\bar{n}_{h,i})\log(2/\tilde{\delta}(\bar{n}_{h,i}))}}_\text{(a)} + \underbrace{\sum_{h=\bar{H}+1}^{H(N)}\sum_{i\in\mathcal{I}_h(N)} \sqrt{T_{h,i}(N)\log(2/\tilde{\delta}(\bar{n}_{h,i}))}}_\text{(b)}\right) \nonumber \,.
\end{align}
\endgroup
In particular, \eqref{eq:gpo.cumulative1} is due to inequality~\eqref{eq:gpo.delta}.

We bound separately the terms (a) and (b). Since $\bar{n}_{h,i}\leq N$, we have
\begin{align*} %\label{eq15}
    \text{(a)} &\leq \sum_{h=0}^{\bar{H}}\sum_{i\in\mathcal{I}_h(N)} \sqrt{\tau_h(N)\log(2/\tilde{\delta}(N))}\\
    &\leq \sum_{h=0}^{\bar{H}}|\mathcal{I}_h(N)| \sqrt{\tau_h(N)\log(2/\tilde{\delta}(N))}.
\end{align*}
Notice that the covering tree is $K$-ary and therefore $|\mathcal{I}_h(N)| \leq K|\mathcal{I}_{h-1}(N)|$. Recall that \gls{hct} only selects a node $(h_n,i_n)$ when its parent is $3\nu\rho^{h_n-1}$-optimal. Therefore, by definition of the near-optimality dimension,
\begin{equation*} %\label{eq16}
|\mathcal{I}_h(N)| \leq| K\mathcal{I}_{h-1}(N)| \leq KC\rho^{-d(h-1)},
\end{equation*}
where $d$ is the near-optimality dimension. As a result, for term (a), we obtain that
\begingroup
\allowdisplaybreaks
\begin{align}
    \text{(a)} & \leq \sum_{h=0}^{\bar{H}} KC\rho^{-d(h-1)} \sqrt{\tau_h(N)\log(2/\tilde{\delta}(N))} \nonumber \\
               & = \sum_{h=0}^{\bar{H}} KC\rho^{-d(h-1)} \sqrt{\frac{c^2\log(2/\tilde{\delta}(N))}{\nu^2}\rho^{-2h}\log(2/\tilde{\delta}(N))} && \label{eq:gpo.term_a} \\
               & = KC\rho^d \frac{c\log(2/\tilde{\delta}(N))}{\nu} \sum_{h=0}^{\bar{H}} \rho^{-h(d+1)} \nonumber \,.
\end{align}
\endgroup
where~\eqref{eq:gpo.term_a} is deduced from inequality~\eqref{eq:gpo.pre}. Consequently, we bound (a) as
\begin{equation} \label{eq17}
    \text{(a)} \leq KC\rho^d \frac{c\log\left(2/\tilde{\delta}(N)\right)}{\nu} \frac{\rho^{-\bar{H}(d+1)}}{1-\rho}\cdot
\end{equation}
We proceed to bound the second term (b). By the Cauchy-Schwarz inequality,
\begin{align*} %\label{eq18}
    \text{(b)} &\leq \sqrt{\sum_{h=\bar{H}+1}^{H(N)}\sum_{i\in\mathcal{I}_h(N)} \log\left(2/\tilde{\delta}\left(\bar{n}_{h,i}\right)\right)} \sqrt{\sum_{h=\bar{H}+1}^{H(N)}\sum_{i\in\mathcal{I}_h(N)} T_{h,i}(N)}\\
    &\leq \sqrt{n\sum_{h=\bar{H}+1}^{H(N)}\sum_{i\in\mathcal{I}_h(N)} \log\left(2/\tilde{\delta}\left(\bar{n}_{h,i}\right)\right)}\,,
\end{align*}
where we trivially bound the second square-root factor by the total number of pulls. Now consider the first square-root factor. Recall that the \gls{hct} algorithm only selects a node when $T_{h,i}(n)\geq\tau_h(n)$ for its parent. We therefore have $T_{h,i}(\tilde{n}_{h,i})\geq\tau_h(\tilde{n}_{h,i})$ and the following sequence of inequalities,
%\vfil\newpage
\begingroup
\allowdisplaybreaks
\begin{align}
    N & = \sum_{h=0}^{H(N)}\sum_{i\in\mathcal{I}_h(N)} T_{h,i}(N) \nonumber \\
      &\geq \sum_{h=0}^{H(N)-1}\sum_{i\in\mathcal{I}_h^+(N)} T_{h,i}(N) \nonumber \\
      & \geq \sum_{h=0}^{H(N)-1}\sum_{i\in\mathcal{I}_h^+(N)} T_{h,i}(\tilde{n}_{h,i}) \label{eq:gpo.decomp1} \\
      & \geq \sum_{h=0}^{H(N)-1}\sum_{i\in\mathcal{I}_h^+(N)} \tau_h(\tilde{n}_{h,i}) \nonumber \\
      & \geq \sum_{h=\bar{H}}^{H(N)-1}\sum_{i\in\mathcal{I}_h^+(N)} \tau_h(\tilde{n}_{h,i}) \nonumber \\
      & = \sum_{h=\bar{H}}^{H(N)-1}\sum_{i\in\mathcal{I}_h^+(N)} \frac{c^2\log(1/\tilde{\delta}(\tilde{n}_{h,i}^+)))}{\nu^2}\rho^{-2h} \nonumber \\
       & \geq \sum_{h=\bar{H}}^{H(N)-1}\sum_{i\in\mathcal{I}_h^+(N)} \frac{c^2\log(1/\tilde{\delta}(\tilde{n}_{h,i}^+)))}{\nu^2}\rho^{-2\bar{H}} \nonumber \\
      & = \frac{c^2\rho^{-2\bar{H}}}{\nu^2} \sum_{h=\bar{H}}^{H(N)-1}\sum_{i\in\mathcal{I}_h^+(N)} \log(1/\tilde{\delta}(\tilde{n}_{h,i}^+))) \nonumber \\
      & = \frac{c^2\rho^{-2\bar{H}}}{\nu^2} \sum_{h=\bar{H}}^{H(N)-1}\sum_{i\in\mathcal{I}_h^+(N)} \log(1/\tilde{\delta}(\left[\max(\bar{n}_{h+1,2i-1},\bar{n}_{h+1,2i})\right]^+)) \label{eq:gpo.decomp2} \\
      & = \frac{c^2\rho^{-2\bar{H}}}{\nu^2} \sum_{h=\bar{H}}^{H(N)-1}\sum_{i\in\mathcal{I}_h^+(N)} \log(1/\tilde{\delta}(\max(\bar{n}_{h+1,2i-1}^+,\bar{n}_{h+1,2i}^+))) \label{eq:gpo.decomp3} \\
      & = \frac{c^2\rho^{-2\bar{H}}}{\nu^2} \sum_{h=\bar{H}}^{H(N)-1}\sum_{i\in\mathcal{I}_h^+(N)} \max(\log(1/\tilde{\delta}(\bar{n}_{h+1,2i-1}^+)),\log(1/\tilde{\delta}(\bar{n}_{h+1,2i}^+))) \nonumber \\
      & \geq \frac{c^2\rho^{-2\bar{H}}}{\nu^2} \sum_{h=\bar{H}}^{H(N)-1}\sum_{i\in\mathcal{I}_h^+(N)} \frac{\log(1/\tilde{\delta}(\bar{n}_{h+1,2i-1}^+))+\log(1/\tilde{\delta}(\bar{n}_{h+1,2i}^+))}{2} \nonumber \\
      & = \frac{c^2\rho^{-2\bar{H}}}{\nu^2} \sum_{h=\bar{H}+1}^{H(N)}\sum_{i\in\mathcal{I}_{h-1}^+(n)} \frac{\log(1/\tilde{\delta}(\bar{n}_{h,2i-1}^+))+\log(1/\tilde{\delta}(\bar{n}_{h,2i}^+))}{2} \label{eq:gpo.decomp4} \\
      & = \frac{c^2\rho^{-2\bar{H}}}{2\nu^2} \sum_{h=\bar{H}+1}^{H(N)}\sum_{i\in\mathcal{I}_h^+(N)} \log(1/\tilde{\delta}(\bar{n}_{h,i}^+)) \nonumber \,.
\end{align}
\endgroup
Note that in~\eqref{eq:gpo.decomp1}, $\tilde{n}_{h,i}$ is well defined for $i\in\mathcal{I}_h^+(N)$. In the above derivation, \eqref{eq:gpo.decomp2} holds since $\tilde{n}_{h,i}=\max(\bar{n}_{h+1,2i-1},\bar{n}_{h+1,2i})$, and~\eqref{eq:gpo.decomp3} holds since $\forall n_1,n_2,\left[\max(n_1,n_2)\right]^+=\max(n_1^+,n_2^+)$. Besides,~\eqref{eq:gpo.decomp4} is simply due to a change of variables. Finally, the last equality relies on the fact that for any $h>0$, $\mathcal{I}_h^+(N)$ covers all the internal nodes at level $h$ and therefore its children cover $\mathcal{I}_{h+1}(N)$. We  thus obtain
\begin{equation} \label{eq19}
\sum_{h=\bar{H}+1}^{H(N)}\sum_{i\in\mathcal{I}_h^+(N)} \log(1/\tilde{\delta}(\bar{n}_{h,i}^+)) \leq \frac{2\nu^2\rho^{2\bar{H}}N}{c^2}\cdot
\end{equation}
On the other hand, we have
\begingroup
\allowdisplaybreaks
\begin{align*}
    \text{(b)} &\leq \sqrt{n\sum_{h=\bar{H}+1}^{H(N)}\sum_{i\in\mathcal{I}_h(N)} \log(2/\tilde{\delta}(\bar{n}_{h,i}))} \\
               &\leq \sqrt{n\sum_{h=\bar{H}+1}^{H(N)}\sum_{i\in\mathcal{I}_h(N)} 2\log(1/\tilde{\delta}(\bar{n}_{h,i}))} \\
               &\leq \sqrt{n\sum_{h=\bar{H}+1}^{H(N)}\sum_{i\in\mathcal{I}_h(N)} 2\log(1/\tilde{\delta}(\bar{n}_{h,i}^+))}\,.
\end{align*}
\endgroup
The last seep holds in the above since $\bar{n}_{h,i}\leq\bar{n}_{h,i}^+$. By plugging (\ref{eq19}) into the expression above, we get
\begin{equation} \label{eq20}
    \text{(b)} \leq \frac{2\nu\rho^{\bar{H}}N}{c}\cdot
\end{equation}
Now if we combine  (\ref{eq20}) with (\ref{eq17}), we  bound  $\tilde{R}_N^{\xi}$ as
\begin{equation} \label{eq21}
    \tilde{R}_N^{\xi} \leq 6\nu\left[KC\rho^d \frac{c^2\log(2/\tilde{\delta}(N))}{\nu^2} \frac{\rho^{-\bar{H}(d+1)}}{1-\rho} + 2\rho^{\bar{H}}N\right]\,.
\end{equation}
We  choose $\bar{H}$ to minimize the above bound by equalizing the two terms in the sum and  obtain
\begin{equation}
    \rho^{\bar{H}} = \left( \frac{KC\rho^dc^2\log(2/\tilde{\delta}(N))}{2n(1-\rho)\nu^2} \right)^{\frac{1}{d+2}}\,,
\end{equation}
which after being plugged into (\ref{eq21}) gives
\begin{equation} \label{eq23}
    \tilde{R}_N^{\xi} \leq 24\nu \left( \frac{KC\rho^dc^2\log(2/\tilde{\delta}(N))}{2(1-\rho)\nu^2} \right)^{\frac{1}{d+2}}N^{\frac{d+1}{d+2}}.
\end{equation}
Finally, combining (\ref{eq23}), (\ref{eq3}), and Lemma~\ref{lemma:gpo.failing}, we obtain
\begin{align*}
    R_N^{\gls{hct}} & \leq \sqrt{N} + \sqrt{2N\log(\frac{4N^2}{\delta}}) + 24\nu\left(\frac{2KC\rho^d}{(1-\rho)^2\nu^2}\right)^{\frac{1}{d+2}}\left(\log\left(\frac{2N}{\delta}\sqrt[8]{\frac{3\nu}{\rho}}\right)\right)^{\frac{1}{d+2}}N^{\frac{d+1}{d+2}} \\
                              & = \sqrt{N} + \sqrt{2N\log(\frac{4N^2}{\delta}}) + 3\left(\frac{2^{3d+7}\nu^d KC\rho^d}{(1-\rho)^{2}}\right)^{\frac{1}{d+2}}\left(\log\left(\frac{2N}{\delta}\sqrt[8]{\frac{3\nu}{\rho}}\right)\right)^{\frac{1}{d+2}}N^{\frac{d+1}{d+2}} \\
                              & \leq 2\sqrt{2N\log(\frac{4N^2}{\delta}}) + 3\left(\frac{2^{3d+7}\nu^d KC\rho^d}{(1-\rho)^{2}}\right)^{\frac{1}{d+2}}\left(\log\left(\frac{2N}{\delta}\sqrt[8]{\frac{3\nu}{\rho}}\right)\right)^{\frac{1}{d+2}}N^{\frac{d+1}{d+2}}
\end{align*}
with probability $1-\delta$.
