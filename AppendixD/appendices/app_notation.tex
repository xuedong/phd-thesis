\section{Notation}\label{app:gpo.notation}

We first fix some constants and introduce some additional notation that are needed for the proof of Theorem~\ref{thm:gpo.hct}.
\begin{itemize}
    \item $c_1 \triangleq (\rho/(3\nu))^{1/8}$, $c \triangleq 2\sqrt{1/(1-\rho)}$
    \item $\forall 1\leq h\leq H(n)$ and $n>0$, $\mathcal{I}_h(n)$ denotes the set of all nodes created by \gls{hct} at level $h$ up to step $t$
    \item $\forall 1\leq h\leq H(n)$ and $n>0$, $\mathcal{I}_h^{+}(n)$ denotes the subset of $\mathcal{I}_h(n)$ which contains only the internal nodes
    \item At each step $n$, $(h_n,i_n)$ denotes the node selected by the algorithm.
    \item $\mathcal{C}_{h,i} \triangleq \{n = 1,\cdots,N : (h_n,i_n) = (h,i)\}$
    \item $\mathcal{C}_{h,i}^{+} \triangleq \underset{j\in\{0,\ldots,K-1\}}{\bigcup} \cC_{h+1,Ki-j}$
    \item $\overline{n}_{h,i} \triangleq \max_{n\in\mathcal{C}_{h,i}} n$ denotes the last time $(h,i)$ has been selected
    \item $\tilde{n}_{h,i} \triangleq \max_{n\in\mathcal{C}_{h,i}^{+}} n$ denotes the last time when one of its children has been selected
    \item $n_{h,i} \triangleq \min\{n: T_{h,i}(N) \geq \tau_h(n)\}$ is the time when $(h,i)$ is expanded
    \item For any $t$, let $y_n\triangleq(r_n,x_n)$ be a random variable, we define the filtration $\cF_n$ as a $\sigma$-algebra generated by $(y_1,\ldots,y_n)$.
\end{itemize}
Another important notion in \gls{hct} is the threshold $\tau_h$ on the number of pulls needed before a node at level $h$ can be expanded.
The threshold $\tau_h$ is chosen such that the two confidence terms in $U_{h,i}$ are roughly equivalent, that is,
\[
    \nu\rho^h \simeq c\sqrt{\frac{\log(1/\tilde{\delta}(n^+))}{\tau_h(n)}}\,.
\]
Therefore, we choose
\[
    \tau_h(n) \triangleq \ceil{\frac{c^2\log(1/\tilde{\delta}(n^+))}{\nu^2}\rho^{-2h}}\,.
\]
Since $n^+$ is defined as $2^{\ceil{\log(n)}}$, we have $n \leq n^+ \leq 2n$. In addition, $\log$ is an increasing function, thus we have
\begin{equation} \label{eq1}
    \frac{c^2}{\nu^2}\rho^{-2h} \leq \frac{c^2 \log(1/\tilde{\delta}(n))}{\nu^2}\rho^{-2h} \leq \tau_h(n) \leq \frac{c^2\log(2/\tilde{\delta}(n))}{\nu^2}\rho^{-2h},
\end{equation}
where the first inequality follows from the fact that $0<\tilde{\delta}(n)\leq1/2$. We begin our analysis by bounding the maximum depth of the trees constructed by \gls{hct}.