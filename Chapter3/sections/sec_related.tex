%!TEX root = ../Chapter3.tex
\section{Two Related Optimality Notions}\label{sec:related}

In the fixed-confidence setting, we aim for building $\delta$-correct strategies, i.e.\ strategies that identify the best arm with high confidence on any problem instance (see Definition~\ref{def:mab.delta}). 

% \begin{definition} A strategy $(I_n,J_n,\tau)$ is $\delta$-correct if for all bandit models $\bmu$ with a unique optimal arm, it holds that $\mathbb{P}_{\bmu}\left[J_{\tau} \neq I^\star\right] \leq \delta$. 
% \end{definition}

Among $\delta$-correct strategies, we seek the one with the smallest sample complexity $\EE{\tau_\delta}$. So far, \TTTS has not been analyzed in terms of sample complexity; \citet{russo2016ttts} focusses on posterior consistency and optimal convergence rates. Interestingly, both the smallest possible sample complexity and the fastest rate of posterior convergence can be expressed in terms of the following quantities.

%\cite{russo2016ttts} proposes an (asymptotic) upper bound on $\Pi_n(\overline{\Theta_{I^\star}}) = 1 - a_{n,I^\star}$, the posterior probability that $I^\star$ is not the best arm. 

%Interestingly, the smallest possible sample complexity and the fastest decay rate for $\Pi_n(\overline{\Theta_{I^\star}})$ can both be expressed in terms of the following quantities.

% For that purpose, we need to introduce two different notions of optimality, that we now formally define and contrast: (1) Bayesian $\beta$-optimality and (2) $\beta$-optimality in the fixed-confidence setting. Before that, we first present some necessary preliminary notions.


\begin{definition} Let $\Sigma_K = \{\bomega : \sum_{k=1}^K \omega_k = 1\}$ and define for all $i\neq I^\star$
\[
    C_i(\omega,\omega') \eqdef \min_{x\in \cI} \ \omega d(\mu_{I^\star};x) + \omega' d(\mu_i;x),
\]
where $d(\mu,\mu')$ is the KL-divergence defined above and $\cI = \R$ in the Gaussian case and $\cI = [0,1]$ in the Bernoulli case. We define
\begin{eqnarray}
    \Gamma^\star &\eqdef& \max_{\bomega \in \Sigma_K}\min_{i\neq I^\star} C_i(\omega_{I^\star},\omega_i),\nonumber\\
    \Gamma_{\beta}^\star &\eqdef& \max_{\substack{\bomega \in \Sigma_K\\\omega_{I^\star}=\beta}}\min_{i\neq I^\star} C_i(\omega_{I^\star},\omega_i).\label{def:GammaBeta}\end{eqnarray}
\end{definition}

The quantity $C_i(\omega_{I^\star},\omega_i)$ can be interpreted as a ``transportation cost''\footnote{for which $W_n(I^\star,i)$ is an empirical counterpart} from the original bandit instance $\bm\mu$ to an alternative instance in which the mean of arm $i$ is larger than that of $I^\star$, when the proportion of samples allocated to each arm is given by the vector $\bomega \in \Sigma_K$. As shown by~\cite{russo2016ttts}, the $\bomega$ that maximizes \eqref{def:GammaBeta} is unique, which allows us to define the $\beta$-\emph{optimal allocation} $\bomega^\beta$ in the following proposition.

\begin{proposition}\label{prop:optim}
\begin{leftbar}[propositionbar]
There is a unique solution $\bomega^\beta$ to the optimization problem \eqref{def:GammaBeta}
satisfying $\omega_{I^\star}^\beta = \beta$, and for all $i,j \neq I^\star$, $C_i(\beta,\omega_i^\beta) = C_j(\beta,\omega_j^\beta)$.
\end{leftbar}
\end{proposition}

\begin{proof}
We handle the existence and the uniqueness separately as below.

\paragraph{Existence:} For any arm $i\neq I^\star$, $C_i$ is a continuous function, so as to $\min_{i\neq I^\star} C_i$. According to the \emph{extreme value theorem}, function $\min_{i\neq I^\star} C_i(\beta,\cdot)$ must attain its maximum over $[0,1]^{K-1}$ which is compact. Suppose that $\bomega^\beta$ is a such maximizer. We thus have
\[
    \Gamma_{\beta}^\star = \max_{\bomega:\omega_{I^\star}=\beta}\min_{i\neq I^\star} C_i(\beta,\omega_i) = \min_{i\neq I^\star} C_i(\beta,\omega_i^\beta).
\]
Let us assume that $\bomega^\beta$ does not verify the second condition, which means there exists some $j\neq I^\star$ such that
\[
    C_j(\beta,\omega_i^\beta) > C_{i^\star}(\beta,\omega_{i^\star}^\beta)\,,
\]
where $i^\star \eqdef \argmin_{i\neq I^\star} C_i(\beta,\omega_i^\beta)$.

Now if we subtract a small quantity $\epsilon>0$, from $C_j(\beta,\omega_j^\beta)$, such that
\[
    \epsilon \leq \frac{C_j(\beta,\omega_j^\beta)-C_{i^\star}(\beta,\omega_{i^\star}^\beta)}{2}\,,
\]
and add $\epsilon/(K-2)$ to $C_i(\beta,\omega_i^\beta)$ for any $i\neq j,I^\star$, we would not change the order of the $C_i(\beta,\omega_i^\beta)$. Therefore, $i^\star$ remains unchanged, however, the new $C_{i^\star}(\beta,\omega_{i^\star}^\beta)$ would be strictly larger than the previous one which contradicts the definition of $\bomega^\beta$.

\paragraph{Uniqueness:} 
We now need to show that the solution is unique. Suppose that two different maximizers $\bomega$ and $\bomega'$ exist, and there exists some $i\neq I^\star$ such that $\omega_i > \omega'_i$. Since $C_i(\beta,\cdot)$ is an strictly increasing function, thus we have $C_i(\beta,\omega_i)>C_i(\beta,\omega'_i)$. By consequence, for any $j\neq i$ and $j'\neq i$,
    \[
        C_j(\beta,\omega_j) = C_i(\beta,\omega_i) > C_i(\beta,\omega'_i) = C_{j'}(\beta,\omega'_{j'}).
    \]
Therefore, for any $j\neq I^\star$ and $j'\neq I^\star$, $\omega_j > \omega'_j$, and
    \[
        \sum_{j\neq I^\star} \omega_j > \sum_{j\neq I^\star} \omega'_j.
    \]
However, we know that $1-\sum_{j\neq I^\star} \omega_j = \omega_{I^\star} =  \beta = \omega'_{I^\star} = 1-\sum_{j\neq I^\star} \omega'_j$, contradiction!

\end{proof}

For models with more than two arms, there is no closed form expression for $\Gamma_\beta^\star$ or $\Gamma^\star$, even for Gaussian bandits  with variance $\sigma^2$ for which we have
\[
    \Gamma_{\beta}^\star = \max_{\bomega:\omega_{I^\star}=\beta}\min_{i\neq I^\star} \frac{(\mu_{I^\star}-\mu_i)^2}{2\sigma^2(1/\omega_i+1/\beta)}.
\]

%We refer the reader to Appendix TBC for pointers on the practical computation of those quantities. 

\paragraph{Bayesian \texorpdfstring{$\beta$}{}-optimality.} \citet{russo2016ttts} proves that  any sampling rule allocating a fraction $\beta$ to the optimal arm ($\Psi_{n,I^\star}/n \rightarrow \beta$) satisfies %$\Pi_n(\overline{\Theta_{I^\star}})
$1-a_{n, I^\star} \geq e^{-n(\Gamma_{\beta}^\star + o(1))}$ (a.s.) for large values of $n$. We define a  \emph{Bayesian $\beta$-optimal} sampling rule as a sampling rule matching this lower bound, i.e. satisfying $\Psi_{n,I^\star}/n \rightarrow \beta$ and $1- a_{n, I^\star} \leq e^{-n(\Gamma_{\beta}^\star + o(1))}$.
%$\Pi_n(\overline{\Theta_{I^\star}}) \leq e^{-n(\Gamma_{\beta}^\star + o(1))}$.

\citet{russo2016ttts} proves that \TTTS with parameter $\beta$ is Bayesian $\beta$-optimal.
However, the result is valid only under strong regularity assumptions, excluding the two practically important cases of Gaussian and Bernoulli bandits. In this paper, we complete the picture by establishing Bayesian $\beta$-optimality for those models in Section~\ref{sec:bayesian}. For the Gaussian bandit, Bayesian $\beta$-optimality was established for \TTEI by~\cite{qin2017ttei} with Gaussian priors, but this remained an open problem for \TTTS.

A fundamental ingredient of these proofs is to establish the convergence of the allocation of measurement effort to the $\beta$-optimal allocation: $\Psi_{n,i}/n \rightarrow \omega_{i}^\beta$ for all $i$, which is equivalent to $T_{n,i}/n \rightarrow \omega_{i}^\beta$ (cf.\ Lemma~\ref{lemma:link}).

\paragraph{\texorpdfstring{$\beta$}{}-optimality in the fixed-confidence setting.} In the fixed confidence setting, the performance of an algorithm is evaluated in terms of sample complexity. A lower bound given by \cite{garivier2016tracknstop} states that any $\delta$-correct strategy satisfies $\EE{\tau_\delta} \geq (\Gamma^\star)^{-1} \ln \left({1}/(3\delta)\right)$. 

Observe that $\Gamma^\star = \max_{\beta \in [0,1]} \Gamma_\beta^\star$. Using the same lower bound techniques, one can also prove that under any $\delta$-correct strategy satisfying $T_{n,I^\star}/n \rightarrow \beta$,
\[\liminf_{\delta \rightarrow 0}\frac{\EE{\tau_\delta}}{\ln(1/\delta)} \geq \frac{1}{\Gamma^\star_\beta}.\]
This motivates the relaxed optimality notion that we introduce in this paper: A BAI strategy is called \emph{asymptotically $\beta$-optimal} if it satisfies 
\[\frac{T_{n,I^\star}}{n}\rightarrow \beta \ \ \ \text{and} \ \ \ \limsup_{\delta \rightarrow 0}\frac{\EE{\tau_\delta}}{\ln(1/\delta)} \leq \frac{1}{\Gamma^\star_\beta}.\]
In the paper, we provide the first sample complexity analysis of a BAI algorithm based on \TTTS (with the stopping and recommendation rules described in Section~\ref{sec:t3c.algorithm}), establishing its asymptotic $\beta$-optimality.

As already observed by \cite{qin2017ttei}, any sampling rule converging to the $\beta$-optimal allocation (i.e.\ satisfying $T_{n,i}/n \rightarrow w_i^\beta$ for all $i$) can be shown to satisfy 
\[
    \limsup_{\delta \rightarrow 0} \tau_\delta/\ln(1/\delta) \leq (\Gamma_\beta^\star)^{-1}
\]
almost surely, when coupled with the Chernoff stopping rule. The fixed confidence optimality that we define above is stronger as it provides guarantees on $\EE{\tau_\delta}$.

% \paragraph{(Parameterized) Optimal weight vector} We first define a characteristic function $C_i$ for each arm $i \neq I^\star$,
% \[
%     C_i(\omega,\omega') \eqdef \min_{x\in\R} \omega d(\mu_{I^\star};x) + \omega' d(\mu_i;x),
% \]
% where $d(x;x')$ denotes the \emph{Kullback Leibler divergence} (\texttt{KL}-divergence) between the observation distributions $p(y|x)$ and $p(y|x')$. This function specifies the effectiveness of distinguishing $\mu_i$ from $\mu_{I^\star}$ using a sampling rule that attributes respectively $\omega$ and $\omega'$ proportion of samples to $I^\star$ and $i$. Intuitively, for a given $\omega$, $C_i(\omega_{I^\star},\omega)\geq C_j(\omega_{I^\star},\omega)$ means that $\mu_i$ is easier to distinguish from $\mu_{I^\star}$ than $\mu_j$. More formally, $C_i$ is a strictly increasing concave function (see~\citealt{russo2016ttts}, Appendix F.1 for more formal properties of these functions).
% 
% It is known that, for a given set of arms $\cA$, a unique optimal effort proportion vector $\bomega^\star$ exists. $\bomega^\star$ is the unique solution to the optimization problem
% \[
%     \Gamma^\star \eqdef \max_{\bomega}\min_{i\neq I^\star} C_i(\omega_{I^\star},\omega_i),
% \]
% and $\Gamma^\star$ represents the fastest error decay rate that any allocation rule can reach~\citep{garivier2016tracknstop,russo2016ttts}.
% 
% It is easy to show that for a given $\beta$, a unique parameterized optimal effort proportion vector $\bomega^\beta$ exists as well (proof given in Appendix~\ref{app:optimal_weight}).
% 
% \begin{restatable}{proposition}{restateoptim}\label{thm:optim}
% There is a unique solution $\bomega^\beta$ to the optimization problem
% \[
%     \Gamma_{\beta}^\star \eqdef \max_{\bomega:\omega_{I^\star}=\beta}\min_{i\neq I^\star} C_i(\beta,\omega_i).
% \]
% that satisfies
% \[
%     \omega_{I^\star}^\beta = \beta,
% \]
% \[
%     \forall i,j\neq I^\star, C_i(\beta,\omega_i^\beta) = C_j(\beta,\omega_j^\beta),
% \]
% \end{restatable}
% 
% In our case, every arm $i$ follows a normal distribution $\cN(\mu_i,\sigma^2)$ with unknown means and known variance, and the priors are assumed to be improper Gaussian priors, thus we have
% \[
%     \Gamma_{\beta}^\star = \max_{\bomega:\omega_{I^\star}=\beta}\min_{i\neq I^\star} \frac{(\mu_{I^\star}-\mu_i)^2}{2\sigma^2(1/\omega_i+1/\beta)}.
% \]
% 
% There is no analytic expression for $\Gamma_{\beta}^\star$, we thus provide an efficient way to approximate $\Gamma_{\beta}^\star$ under Gaussian priors as well as in the general case in Appendix~\ref{app:rate}.


