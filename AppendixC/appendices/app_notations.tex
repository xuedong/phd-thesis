%!TEX root = ../AppendixC.tex
\section{Notation}\label{app:lgc.notations}

\begin{table}[ht]
	\centering
	\caption{Table of notation for Chapter~\ref{CHAP:LGC}.}
	\begin{tabular}{@{}l|l@{}}
		\toprule
		\thead{Notation} & \thead{Meaning} \\ \midrule
		$\Theta$ & set of parameters \\
        $M$ & upper bound on the norm of $\theta$\\
		$\cX$ & finite set or arms  \\
        $K$ & number of arms \\
        $\cY$ & transductive set \\
        $B$ & number of elements in the transductive set \\
        $\cI$ & finite set of answers \\
        $A$ & number of answers \\
        $L$ & upper bound on the norms of the arms\\
        $\btheta$ & parameter in $\Theta$ \\
        $\hbx_n$ & arm pulled at time $n$ \\
        $T_n^\bx = \sum_{t=1}^n \ind_{\{\hbx_t = \bx\}}$ & number of draws of arm $\bx$ up to time $n$\\
        $T_{n,i}$ & number of draws of arm indexed $i$ up to time $n$\\
        $\bT_n =(T_n^\bx)_{\bx\in\cX}$ & vector of number of draws\\
        $T_n^{\bx,i} = \sum_{t=1}^n \ind_{\{\hbx_t = \bx, i_t = i\}}$ & number of draws of arm $\bx$ for a given answer $i$\\
        $\lambda$ & regularization parameter\\
        $\hbtheta_n^\lambda$ & regularized least square estimate\\
		\bottomrule
	\end{tabular}
\end{table}
